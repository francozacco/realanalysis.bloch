\documentclass[11pt]{article}
\usepackage{amssymb}
\usepackage{amsthm}
\usepackage{enumitem}
\usepackage{amsmath}
\usepackage{bm}
\usepackage{adjustbox}
\usepackage{mathrsfs}
\usepackage{graphicx}
\usepackage{siunitx}
\usepackage[mathscr]{euscript}

\title{\textbf{Solved selected problems of Real Numbers and Real Analysis - Bloch}}
\author{Franco Zacco}
\date{}

\addtolength{\topmargin}{-3cm}
\addtolength{\textheight}{3cm}

\newcommand{\N}{\mathbb{N}}
\newcommand{\Z}{\mathbb{Z}}
\newcommand{\Q}{\mathbb{Q}}

\theoremstyle{definition}
\newtheorem*{solution*}{Solution}

\begin{document}
\maketitle
\thispagestyle{empty}

\section*{Chapter 1 - Construction of the Real Numbers}

	\begin{proof}{\textbf{1.2.1}}
        To prove the uniqueness of $\cdot:\N\times\N \rightarrow \N$, we
        suppose that there are two operations $\cdot$ and $\odot$ on $\N$ that
        satisfy the two properties of the theorem. Let
        $$G = \{x\in\N~|~n\cdot x = n \odot x\text{ for all }n \in\N \}$$
        we want to prove that $G = \N$, which will imply that $\cdot$ and $\odot$
        are the same operation. It is clear that $G \subseteq \N$. By part (a)
        applied to each of $\cdot$ and $\odot$ we see that
        $n\cdot 1 = n = n \odot 1$ for all $n \in \N$ and then $1 \in G$.
        
        Now let $q \in G$ so $n \cdot q = n \odot q$ and then from part (b) we
        have that $n \cdot s(q) = (n\cdot q) + n = (n\odot q) + n = n\odot s(q)$
        hence $s(q) \in G$.

        Finally, we use part (c) of Peano Postulates to conclude that $G = \N$
    \end{proof}
	\begin{proof}{\textbf{1.2.2}}
        \begin{itemize}
        \item [(2)] Let 
        $$G = \{c \in \N ~|~ (a+b)+c=a+(b+c) \text{ for all }a,b\in\N\}$$
        We will show that $G = \N$ which will imply the desired result. Clearly
        $G\subseteq\N$. To show that $1 \in G$, let $j, k \in \N$ so
        $(j+k)+1=s(j+k)=j+s(k)=j+(k+1)$ then $1 \in G$, where we used from
        Theorem 1.2.5 that $s(n+m) = n + s(m)$.

        Now let $r \in G$. Let $j,k \in G$ and suppose $(j+k)+s(r) = j + (k+s(r))$
        by Theorem 1.2.5 we know that $(j+k)+s(r)=s((j+k)+r)$ and since
        $r \in G$ then $s((j+k)+r)=s(j+(k+r))$ then $s(j+(k+r))=j+s(k+r)=j+(k+s(r))$
        therefore $s(r) \in G$ and $G =\N$.
        \item [(3)] Let
        $$H = \{a \in\N ~|~ a+1=1+a=s(a)\}$$
        We will show that $H = \N$ which will imply the desired result. Clearly
        $H\subseteq\N$. If $a=1$ replacing we have that $1+1=1+1=s(1)$ then
        $1\in H$.
        
        Now let $r \in H$ and suppose $s(r)+1=1+s(r)$ then
        \begin{align*}
            s(r)+1 &= (r+1)+1 &\text{because part (a) of Theorem 1.2.5}\\
                &= (1+r)+1 &\text{because } r \in H \\
                &= 1+(r+1) &\text{because } a+(b+c) = (a+b)+c \\
                &= 1+s(r) &\text{because part (a) of Theorem 1.2.5}
        \end{align*}
        Therefore $H = \N$.
        \item [(4)] Let
        $$G = \{a\in\N ~|~ a+b=b+a \text{ for all }b\in\N\}$$
        We will show that $G = \N$ which will imply the desired result. It is
        clear that $G\subseteq\N$. To show that $1 \in G$, let $j\in \N$ so
        $1+j=j+1$ because we proved that in part (3) then $1 \in G$.

        Now let $r\in G$ and suppose that $j +s(r) = s(r)+j$ then
        \begin{align*}
            j + s(r) &= s(j+r) &\text{because part (b) of Theorem 1.2.5} \\
                &= s(r+j) &\text{because }r\in G \\
                &= r + s(j) &\text{because part (b) of Theorem 1.2.5} \\
                &= r + (j + 1) &\text{because part (a) of Theorem 1.2.5} \\
                &= r + (1 + j) &\text{because part (3)} \\
                &= (r + 1) + j &\text{because associative law for addition} \\
                &= s(r) + j &\text{because part (a) of Theorem 1.2.5}
        \end{align*}
        Therefore $G = \N$.
        \item [(7)] Let
        $$G = \{a \in N ~|~ 1\cdot a = a \cdot 1 = a\}$$
        We will show that $G = \N$ which will imply the desired result. It is
        clear that $G\subseteq\N$. If $a=1$ replacing we have that
        $1\cdot 1 = 1\cdot 1 = 1$ because of part (a) in Theorem 1.2.6 then
        $1 \in G$.

        Now let $r \in G$ and suppose $s(r)\cdot 1 = 1 \cdot s(r) = s(r)$ then
        \begin{align*}
            s(r) \cdot 1 &= s(r) &\text{because part(a) of Theorem 1.2.6}\\
                &= r + 1 &\text{because part(a) of Theorem 1.2.5}\\
                &= (r \cdot 1) + 1 &\text{because }r \in G\\
                &= (1 \cdot r) + 1  &\text{because }r \in G\\
                &= 1 \cdot s(r) &\text{because part (b) of Theorem 1.2.6}
        \end{align*}
        Therefore $G = \N$.
\cleardoublepage
        \item [(8)] Let
        $$H = \{c \in \N ~|~ (a+b)c=ac+bc \text{ for all }a,b \in \N\}$$
        We want to show that $H = \N$ which will imply the desired result. It 
        is clear that $H\subseteq\N$. To show that $1 \in H$ let $j,k \in \N$
        then $(j+k)\cdot 1 = j+k = j\cdot 1 +k \cdot 1$ because what we proved
        in part (7), then $1 \in H$.

        Now let $r \in H$, let $j,k \in \N$ and suppose
        $(j+k)\cdot s(r) = j\cdot s(r) + k\cdot s(r)$ then
        \begin{align*}
            (j+k)\cdot s(r) &= ((j+k)\cdot r) + (j+k) &\text{because part (b) of Theorem 1.2.6}\\
                &= (j\cdot r + k \cdot r) + (j+k) &\text{because }r \in H\\
                &= (j\cdot r + j) + (k\cdot r + k) &\text{because Commutative and Associative law} \\
                &= j\cdot s(r) + k \cdot s(r) &\text{because part (b) of Theorem 1.2.6}
        \end{align*}
        Therefore $s(r) \in H$ and $H = \N$.
        \end{itemize}
        \item [(9)] Let
        $$H = \{a \in \N ~|~ ab=ba \text{ for all }b \in \N\}$$
        We want to show that $H = \N$ which will imply the desired result. It 
        is clear that $H\subseteq\N$. Also $1 \in H$ because what we proved in
        part (7).

        Now let $r \in H$, let $k \in \N$ and suppose $s(r)\cdot k = k \cdot s(r)$
        then
        \begin{align*}
            s(r) \cdot k &= (r+1) \cdot k & \text{because part (a) of Theorem 1.2.5} \\
                &= r\cdot k + 1\cdot k & \text{because Distributive law} \\
                &= k\cdot r + k & \text{because }r \in H \\
                &= k \cdot s(r) & \text{because part (b) of Theorem 1.2.6}
        \end{align*}
        Therefore $s(r) \in H$ and $H = \N$.
        \item [(10)] Let
        $$H = \{c \in \N ~|~ c(a+b)=ca+cb \text{ for all }a,b \in \N\}$$
        We want to show that $H = \N$ which will imply the desired result. It 
        is clear that $H\subseteq\N$. Let $j,k \in \N$, if $c=1$ then
        $1\cdot(j+k) = j+k = 1\cdot j + 1\cdot k$ so $1 \in H$.

        Now let $r \in H$ and lets suppose that
        $s(r)\cdot(j+k)=s(r)\cdot j + s(r) \cdot k$ then
        \begin{align*}
            s(r)\cdot(j+k) &= (j+k)\cdot s(r) & \text{because part (9)}\\
                &= j\cdot s(r) + k\cdot s(r) & \text{because right-hand side Distributive law}\\
                &= s(r) \cdot j + s(r) \cdot k & \text{because part (9)}
        \end{align*}
        Therefore $s(r) \in H$ and $H = \N$.
\cleardoublepage
        \item [(11)] Let
        $$H = \{c \in \N ~|~ (ab)c=a(bc) \text{ for all }a,b \in \N\}$$
        We want to show that $H = \N$ which will imply the desired result. It 
        is clear that $H\subseteq\N$. Let $j,k \in \N$, if $c = 1$ then
        $(j\cdot k)\cdot 1 = j\cdot k = j\cdot (k) = j\cdot (k\cdot 1)$ then $1 \in H$.
        
        Now let $r \in H$ and suppose $(j\cdot k)\cdot s(r) = j\cdot(k\cdot s(r))$
        then
        \begin{align*}
            (j \cdot k) \cdot s(r) &= ((j \cdot k) \cdot r ) + (j \cdot k) & \text{because part (b) of Theorem 1.2.6}\\
                &= (j \cdot (k \cdot r)) + (j \cdot k) & \text{because }r \in H \\
                &= j \cdot ((k\cdot r) + k) & \text{because Distributive law} \\
                &= j \cdot (k \cdot s(r)) & \text{because part (b) of Theorem 1.2.6}
        \end{align*}
        Therefore $s(r) \in H$ and $H = \N$.
        \item [(13)] Let $ab=1$ there are a set of cases that we should check
        \begin{itemize}
            \item if $a = 1$ and $b \neq 1$ then $1 \cdot b = b$ but we said
            that $ab = 1$ then $b$ must be equal to 1 which is a contradiction.
            \item if $a \neq 1$ and $b = 1$ then $a \cdot 1 = a$ but we said
            that $ab = 1$ then $a$ must be equal to 1 which is a contradiction.
            \item if $a \neq 1$ and $b \neq 1$ then because of Lemma 1.2.3
            there is a unique $t \in \N$ such that $b= s(t)$ so 
            $a \cdot b = a \cdot s(t) = a \cdot t + a = 1$ and because of part
            (5) this can't be true.
        \end{itemize}        
        Now let $a=b=1$ then $a\cdot b=1 \cdot 1 = 1$ which is what we wanted.
    \end{proof}
	\begin{proof}{\textbf{1.2.3}} Let $p_1, p_2 \in\N$ where $p_1 \neq p_2$
        such that $a +p_1 = b$ and $a + p_2 = b$ then $a + p_1 = a + p_2$ and
        because of the Cancellation law we have that $p_1 = p_2$ which is a
        contradiction. Therefore there is a unique $p \in \N$ such that
        $a + p = b$. 
    \end{proof}
	\begin{proof}{\textbf{1.2.4}}
        \begin{itemize}
        \item [(1)] Since $a = a$ then by definition of the operation $\leq$
        is clear that $a \leq a$.

        Now let $a < a$ then there is a $p\in\N$ such that $a = a + p$ but
        this is not possible because of part (6) of Theorem 1.2.7 then it's
        a contradiction and therefore $a \nless a$.

        Let $b = a + 1$ then if $a < b$ there is a $p \in \N$ such that
        $b = a + p$ but then $a + 1 = a + p$ and then $p = 1$, so we found
        $p = 1$ for which $a < b = a + 1$.
\cleardoublepage
        \item [(3)] If $a < b$ and $b < c$ then there is a $p \in \N$ and a
        $q \in \N$ such that $b = a + p$ and $c = b + q$ so replacing variable
        $b$ we have that $c = (a + p) + q = a + (p + q)$ now naming $k = p + q$
        we have $c = a + k$ and then by definition $a < c$.
        
        If $a \leq b$ and $b <c$ then either $a < b$ or $a = b$, the first case
        was already proven so we focus on the second one. We also have that
        $b < c$ then by definition there is a $p \in \N$ such that $c = b + p$
        but $a = b$ then replacing $c = a + p$ and by definition $a<c$.

        If now, $a < b$ and $b \leq c$ we have that either $b < c$ or $b = c$
        the first case was already proven so we focus on the second one.
        Given that $a < b$ there is a $p \in \N$ such that $b = a +p$ but if
        $b = c$ then $c = a + p$ which by definition says that $a < c$.

        Finally, if $a \leq b$ and $b \leq c$ then either $a<b$ or $a=b$ and
        $b<c$ or $b =c$, the last combination we have to prove is the case
        where $a=b$ and $b=c$ then $a=b=c$ so it's clear that $a \leq c$.
        \item [(4)] Let $a < b$ then by definition there is a $p \in \N$ such
        that $b = a + p$ then $b + c = (a + p) + c$ because of part (1) of
        Theorem 1.2.7, and by the Commutative and Associative law we have that
        $b + c = (a + c) + p$ which by definition says that $a + c < b + c$.

        If $a + c < b +c$ then by definition there is a $p \in \N$ such that
        $b + c = (a + c) + p$ and by the Commutative and Associative law we
        have that $b + c = (a + p) + c$ and because of part (1) of Theorem
        1.2.7 we have that $b = a + p$ which by definition says that $a < b$.
        \item [(5)] Let $a<b$ then by definition there is a $p \in \N$ such
        that $b = a + p$ and because of part (12) of Theorem 1.2.7 we have that
        $bc = (a+p)c = ac + pc$ where we also applied the Distributive law
        now naming $k = pc$ we have that $bc = ac + k$ which by definition
        means that $ac < bc$.

        Let $ac < bc$ and suppose $a\nless b$ then by the Trichotomy law either
        $a>b$ or $a=b$. If $a>b$ then because what we proved $ac>bc$ which is
        a contradiction to $ac < bc$. If $a = b$ then $ac = bc$ because of part
        (12) of Theorem 1.2.7. and it's another contradiction to the fact that
        $ac < bc$. Then must happen that $a<b$.
        \item [(11)] Let $a < b$ and suppose $b < a+1$ then $a < b < a + 1$ but
        this cannot happen because of part (9) so it must happen because of the
        Trichotomy law that $a + 1 \leq b$.

        Now let $a + 1 \leq b$ and suppose that $a = b$ then $a + 1 \leq b = a$
        which cannot be because of part (1) of this Theorem, so lets suppose
        that $a > b$ then $a + 1 < a$ because of part (3) of this Theorem but
        that cannot be true because of part (1) of this Theorem. Therefore
        it must be that $a < b$.
        \end{itemize}
    \end{proof}
    \begin{proof}{\textbf{1.2.5}}
        Let $a+a = b+b$ then because of part (7) of Theorem 1.2.7 we can write
        that $1\cdot a + 1 \cdot a = 1 \cdot b + 1 \cdot b$ and because of the
        Distributive law we have that $a\cdot (1 + 1) = b\cdot (1 + 1)$ if we
        name $c = 1 + 1$ then $ac=bc$ and because of part (12) of Theorem 1.2.7
        we have that $a = b$.
    \end{proof}
    \begin{proof}{\textbf{1.2.6}}
        Let
        $$H = \{n \in \N ~|~1 \leq n \leq b\} \cup\{n \in \N ~|~ b+1 \leq n\}$$
        We want to show that $H = \N$. It is clear that $H\subseteq\N$. Because
        $1 \in \{n \in \N ~|~1 \leq n \leq b\}$ by definition then $1 \in H$.\\
        Now let $r \in H$, we want to show that $r+1 \in H$, if $r=b$ then
        $$r+1=b+1 \in \{n \in \N ~|~ b+1 \leq n\}$$
        if $r < b$ then because of part (11) of Theorem 1.2.9 $r+1 \leq b$ then
        $$r+1 \in \{n \in \N ~|~1 \leq n \leq b\}$$
        if $b < r$ then because of part (11) or Theorem 1.2.9 $b+1 \leq r$
        also because of part (1) of Theorem 1.2.9 $r < r+1$ then because
        of part (3) of Theorem 1.2.9 we have that $b+1 < r+1$ then
        $$r+1 \in \{n \in \N ~|~ b+1 \leq n\}$$
        Therefore $r+1 \in H$ and $H = \N$.\\
        Now Let
        $$G = \{n \in \N ~|~1 \leq n \leq b\} \cap\{n \in \N ~|~ b+1 \leq n\}$$
        Suppose there is an $r\in G$. We will derive a contradiction.
        Then it must happen that $1\leq r \leq b$ and that $b +1 \leq r$ but
        then because of part (3) of Theorem 1.2.9 it must happen that 
        $b+1 \leq b$ which is a contradiction to the part (6) of Theorem 1.2.9.
        Therefore there is no $r \in G$.
    \end{proof}
    \begin{proof}{\textbf{1.2.7}}
        \begin{itemize}
        \item [(1)] Let
        $$H = \{n \in \N ~|~ a+n \in A \text{ for all }a\in A\}$$
        We want to show that $H = \N$ which will imply the desired result. It 
        is clear that $H\subseteq\N$. To show that $1 \in H$ let $b \in A$ then
        $b + 1 \in A$ by definition of $A$ and then $1 \in H$.

        Now let $r \in H$ then $b + r \in A$ for some $b \in A$ and by
        definition of $A$ we have that $(b + r) + 1 = b + (r + 1)\in A$ then
        $r +1 \in H$ and therefore $H = \N$.
        \item [(2)] Let $H = \{x \in \N ~|~ x \geq a\}$ and let $r \in H$ then
        $r \geq a$ so it must happen that $r = a$ or $r > a $ in the first
        case it is clear that $r = a \in A$ in the second case by definition
        there is a $p \in \N$ such that $r = a + p$ and we know because of the
        part (1) that $a + p \in A$. Therefore it must happen that
        $H \subseteq A$.
        \end{itemize}
    \end{proof}
    \begin{proof}{\textbf{1.2.8}}
        We want to prove that there is an inverse function for $f$. We have a
        set $\N'$ with an element $1'\in\N'$ and a function
        $s':\N'\rightarrow \N'$ that satisfy the Peano Postulates so we could
        say because Theorem 1.2.4 that there is a function
        $g:\N' \rightarrow \N$ such that $g(1') = 1$ and $g \circ s' = s \circ g$.
        Now we have to check that $g$ is the inverse of $f$.\\
        Let
        $$G = \{n \in \N ~|~ g(f(n)) = n\}$$
        We want to show that $G = \N$. It  is clear that $G\subseteq\N$.
        To show that $1 \in G$ we do $g(f(1)) = g(1') = 1$ which means that
        $1 \in G$.\\
        Now Let $r \in G$ we want to show that $r+1 \in G$ then
        \begin{align*}
            g(f(r+1)) &= g(f(s(r))) & \text{by definition of }r+1\\
                &= g(s'(f(r))) & \text{because }f \circ s = s' \circ f\\
                &= s(g(f(r))) & \text{because }g \circ s' = s \circ g\\
                &= s(r) = r + 1 & \text{because }r \in G
        \end{align*}
        Therefore $r+1 \in G$ and $G = \N$.
        In the same way, let
        $$H = \{n \in \N' ~|~ f(g(n)) = n\}$$ 
        We want to show that $H = \N'$. It  is clear that $H\subseteq\N'$.
        To show that $1' \in H$ we do $f(g(1')) = f(1) = 1'$ which means that
        $1' \in H$.\\
        Now let $r' \in H$ we want to show that $r' +1 \in H$ then
        \begin{align*}
            f(g(r'+1)) &= f(g(s'(r'))) & \text{by definition of }r'+1\\
                &= f(s(g(r'))) & \text{because }g \circ s' = s \circ g\\
                &= s'(f(g(r'))) & \text{because }f \circ s = s' \circ f\\
                &= s'(r') = r' + 1 & \text{because }r' \in H
        \end{align*}
        Therefore $r'+1 \in H$ and $H = \N'$.\\
        Finally, $g$ is the inverse of $f$ hence $f$ is bijective, which is
        what we wanted.
    \end{proof}
    \begin{proof}{\textbf{1.3.1}}
        \begin{itemize}
            \item [(1)] We want to prove that $\approx$ is an equivalence relation so we
            must prove that it is reflexive, symmetric and transitive. Let
            $(a,b),(c,d) \in \N \times \N$. Given that $a^2b=a^2b$ then
            $(a,b) \approx (a,b)$. Therefore $\approx$ is reflexive. \\
            Suppose that $(a,b) \approx (c,d)$ then $a^2d=c^2b$ but also $c^2b=a^2d$
            hence $(c,d) \approx (a,b)$ and therefore $\approx$ is symmetric. \\
            Now suppose that also $(e,f) \in \N \times \N$ and $(c,d) \approx (e,f)$ 
            then $c^2f = e^2d$ but we know that $a^2d=c^2b$ multiplying this last
            equation on both sides with $f$ and doing a few re-arrangements we have that
            $a^2df = c^2fb = e^2db$ then $a^2f=e^2b$ which means that
            $(a,b) \approx (e,f)$. Therefore $\approx$ is transitive.\\
            Finally since $\approx$ is reflexive, symmetric and transitive then
            $\approx$ is an equivalence relation on $\N \times \N$.
            \item [(2)] The elements of the equivalence class $[(2,3)]$ are
            $$[(2,3)] = \{(x,y)\in\N\times\N~|~ 4y = (x^2)3\}$$
        \end{itemize}
    \end{proof}
    \begin{proof}{\textbf{1.3.2}}
            We want to complete the proof by showing that $\sim$ is transitive. Let
            $(e,f) \in \N \times \N$ and $(c,d) \sim (e,f)$ then $c+f = d+e$ also we
            know that $a+d = b+c$ and adding to both sides $f$ we get that
            $a+d+f = b+c+f = b+d+e$ then $a+f = b+e$ because of Theorem 1.2.7 part (1).
            Therefore $(a,b)\sim(e,f)$.\\
            Finally since $\sim$ is reflexive, symmetric and transitive then
            $\sim$ is an equivalence relation on $\N \times \N$.
    \end{proof}
    \begin{proof}{\textbf{1.3.3}}
        Let's prove first that $-$ is well-defined for $\Z$. Let
        $(a,b),(x,y) \in \N \times \N$ and suppose that $[(a,b)] = [(x,y)]$ then
        $(a,b)\sim(x,y)$ so $a+y = b+x$ but if we write $b+x=a+y$ we have that
        $(b,a) \sim (y,x)$ therefore $-[(a,b)]=[(b,a)]=[(y,x)]=-[(x,y)]$.\\
        Now let's prove that $\cdot$ is well-defined for $\Z$. Let
        $(a,b),(c,d),(x,y),(z,w) \in \N \times \N$ and suppose $[(a,b)] = [(x,y)]$ and
        $[(c,d)] = [(z,w)]$ so by hypothesis $(a,b) \sim (x,y)$ and $(c,d) \sim (z,w)$
        then $a+y=b+x$ and $c+w=d+z$.
        Taking into account this let's do
        \begin{align*}
            &(ac + bd + xw + yz) + (xc + yc + xd + yd) = \\
            &= c(a + y + x) + d(b + x + y) + xw + yz \\
            &= c(b + x + x) + d(a + y + y) + xw + yz \\
            &= bc + xc + xc + ad + yd + yd + xw + yz \\
            &= ad + bc + x(c + c + w) + y(d + d + z) \\
            &= ad + bc + x(c + d + z) + y(d + c + w) \\
            &= (ad + bc + xz + yw) + (xc + yc + xd + yd)
        \end{align*}
        which proves that $ac + bd + xw + yz  = ad + bc + xz + yw$ and therefore
        $[(ac+bd,ad+bc)] = [(xz+yw,xw+yz)]$.
    \end{proof}
    \begin{proof}{\textbf{1.3.4}}
    \begin{itemize}
        \item [(1)]
        ($\rightarrow$) If $[(a,b)]=\hat{0}$ then $[(a,b)] = [(1,1)]$ because of the
        definition of $\hat{0}$ then $(a,b) \sim (1,1)$ so $a+1=b+1$ because of the
        Cancellation Law of $\N$ we have that $a=b$.\\
        ($\leftarrow$) If $a=b$ then adding to both sides $1$ we have that $a+1=b+1$
        then by the Definition 1.3.1 $(a,b)\sim(1,1)$ therefore $[(a,b)]=[(1,1)]$.
        \item [(2)]
        ($\rightarrow$) If $[(a,b)]=\hat{1}$ then $[(a,b)] = [(1+1,1)]$ because of the
        definition of $\hat{1}$ then $(a,b) \sim (1+1,1)$ so $a+1=(b+1)+1$ because of
        the Cancellation Law of $\N$ we have that $a=b+1$.\\
        ($\leftarrow$) If $a=b+1$ then adding to both sides $1$ we have that
        $a+1=(b+1)+1=b+(1+1)$ then by the Definition 1.3.1 $(a,b)\sim(1+1,1)$
        therefore $[(a,b)]=[(1+1,1)]$.
        \item [(3)]
        First let's prove that $[(a,b)]=[(n,1)]$ for some $n \in \N$ such that
        $n \neq 1$ is and only if $a>b$.\\
        ($\rightarrow$) Let $[(a,b)]=[(n,1)]$ for some $n \in \N$ where $n \neq 1$ then
        $a+1=b+n$ and given that $n\in\N$ and $n \neq 1$ then $n$ can be written as
        $n = q +1$ for some $q\in\N$ then $a+1=b+q+1$ and by the Cancellation law of
        $\N$ we have that $a=b+q$ which by definition means that $a>b$.\\
        ($\leftarrow$) Let $a>b$ by definition this means that $a = b +m$ where $m\in\N$
        then if we add $1$ to both sides of the equation we have that
        $a + 1 = b + m + 1$ then by naming $n = m + 1$ where $n \in \N$ we have that
        $a+1=b+n$ and therefore $[(a,b)]=[(n,1)]$ where $n \neq 1$.\\
        Now Let's prove that $a>b$ if and only if $[(a,b)]>\hat{0}$\\
        ($\rightarrow$) Let $a>b$  if we add on both sides of the equation $1$ we get
        that $a+1>b+1$ then this means that $[(a,b)]>\hat{0}=[(1,1)]$.\\
        ($\leftarrow$) Let $[(a,b)]>\hat{0}=[(1,1)]$ then $a+1>b+1$ and because of
        Theorem 1.2.9 part (4) we have that $a>b$.
        \item [(4)]
        First let's prove that $[(a,b)]=[(1,m)]$ for some $m \in \N$ such that
        $m \neq 1$ if and only if $a<b$.\\
        ($\rightarrow$) Let $[(a,b)]=[(1,m)]$ for some $m \in \N$ where $m \neq 1$ then
        $a+m=b+1$ and given that $m\in\N$ and $m \neq 1$ then $m$ can be written as
        $m = q +1$ for some $q\in\N$ then $a+q+1=b+1$ and by the Cancellation law of
        $\N$ we have that $a+q=b$ which by definition means that $a<b$.\\
        ($\leftarrow$) Let $a<b$ by definition this means that $b = a +n$ where $n\in\N$
        then if we add $1$ to both sides of the equation we have that
        $b + 1 = a + n + 1$ then by naming $m = n + 1$ where $m \in \N$ we have that
        $a+m=b+1$ and therefore $[(a,b)]=[(1,m)]$ where $m \neq 1$.\\
        Now Let's prove that $a<b$ if and only if $[(a,b)]<\hat{0}$\\
        ($\rightarrow$) Let $a<b$  if we add on both sides of the equation $1$ we get
        that $a+1<b+1$ then this means that $[(a,b)]<\hat{0}=[(1,1)]$.\\
        ($\leftarrow$) Let $[(a,b)]<\hat{0}=[(1,1)]$ then $a+1<b+1$ and because of
        Theorem 1.2.9 part (4) we have that $a<b$.
    \end{itemize}
    \end{proof}
    \begin{proof}{\textbf{1.3.5}}
    \begin{itemize}
    \item [(1)]
    Using the definition of addition of integers we see that
    \begin{align*}
        (x+y)+z &= ([(a,b)]+[(c,d)])+[(e,f)] \\
                &= [(a+c,b+d)] + [(e,f)] \\
                &= [((a+c)+e,(b+d)+f)] \\
                &= [(a+(c+e),b+(d+f))] \\
                &= [(a,b)] + [(c+d,d+f)] \\
                &= [(a,b)] + ([(c,d)] + [(e,f)]) \\
                &= x+(y+z)
    \end{align*}
    where the middle equality holds because of the Associative law of $\N$.
    \item [(3)]
    We want to prove that $x+\hat{0} = x$ by arriving to a contradiction. Let us suppose
    $x + \hat{0} \neq x$ then this means that $[(a,b)]+[(1,1)] = [(c,d)]$ where
    $[(c,d)] \neq [(a,b)]$ then $[(a+1,b+1)] = [(c,d)]$ so $a+1+d=b+1+c$ and by the
    Cancellation of $\N$ we have that $a+d=b+c$ then $[(a,b)]=[(c,d)]$ which is a
    contradiction and therefore $x+\hat{0} = x$.
    \item [(4)]
    Let $x=[(a,b)]$ then from the equation $(a+b)+1=(b+a)+1$ we have that
    $[(a+b,b+a)]=[(1,1)]$ and thus $[(a,b)]+[(b,a)] = [(1,1)]$ therefore $x+(-x)=\hat{0}$. 
    \item [(5)]
    Let $x = [(a,b)]$, $y = [(c,d)]$ and $z = [(e,f)]$ then
    \begin{align*}
        (xy)z &= ([(a,b)]\cdot[(c,d)])\cdot[(e,f)] \\
              &= [(ac + bd,ad + bc)] \cdot[(e,f)] \\
              &= [((ac + bd)e+(ad + bc)f, (ac + bd)f+(ad + bc)e)] \\
              &= [(ace + bde+adf + bcf, acf + bdf+ ade + bce)] \\
              &= [(a(ce + df) + b(de + cf), a(cf + de) + b(df + ce))] \\
              &= [(a,b)] \cdot [(ce + df, de + cf)] \\
              &= [(a,b)] \cdot ([(c,d)] \cdot [(e,f)]) = x(yz)
    \end{align*}
    \item [(6)]
    Let $x = [(a,b)]$ and $y = [(c,d)]$ then
    \begin{align*}
        xy &= [(a,b)]\cdot[(c,d)] \\
           &= [(ac+bd,ad+bc)] \\
           &= [(ca+db,cb+da)] \\
           &= [(c,d)]\cdot[(a,b)] = yx    
    \end{align*}
    Where the middle equality is true because of the Commutative Law for Addition and
    Multiplication.
    \item[(7)]
    Let $a,b\in\N$ then the equation $a(1+1)+b(1)+b=b(1+1)+a(1)+a$ is true and if we
    let $x=[(a,b)]$ then the equation means that
    $$[(a(1+1) + b(1),b(1+1) + a(1))] = [(a,b)]$$
    then because of the definition of the $\cdot$ operation we have that
    $$[(a,b)] \cdot [(1+1,1)] = [(a,b)]$$
    therefore $x \cdot \hat{1} = x$.
    \item [(8)]
    Let $x = [(a,b)]$, $y = [(c,d)]$ and $z = [(e,f)]$ then
    \begin{align*}
        x(y+z) &= [(a,b)]\cdot([(c,d)] + [(e,f)]) \\
               &= [(a,b)]\cdot[(c+e,d+f)] \\
               &= [(a(c+e)+b(d+f),a(d+f)+b(c+e))] \\
               &= [(ac+ae+bd+bf,ad+af+bc+be)] \\
               &= [((ac+bd)+(ae+bf),(ad+bc)+(af+be))] \\
               &= [(ac+bd,ad+bc)] + [(ae+bf,af+be)] \\
               &= [(a,b)]\cdot[(c,d)] + [(a,b)]\cdot[(e,f)] = xy + xz
    \end{align*}
    \item [(10)]
    Let us first prove that there is no way that two of $x<y$, $x=y$ or $x>y$ can be
    true at the same time. \\
    Let $x = [(a,b)]$ and $y = [(c,d)]$ and let's suppose that $x<y$ and $x=y$ are both
    true then $x<x$ which means that $[(a,b)]<[(a,b)]$ and thus $a+b<b+a$
    by the Cancellation law $a<a$ but that is a contradiction to Theorem 1.2.9 part (1).\\
    In the same way let's suppose that $x>y$ and $x=y$ are both true then $x>x$, which
    leads to the same result as before which is a contradition.\\
    Finally, let's suppose that $x<y$ and $y<x$ are both true then from the first
    inequality we have that $[(a,b)]<[(c,d)]$ then $a+d<b+c$, and from the last inequality
    we have that $[(c,d)]<[(a,b)]$ then $c+b<a+d$ and by applying the Transitive law of
    $\N$ we have that $a+d<a+d$ thus $a<a$ and we have already proven that this is a
    contradiction.\\
    Therefore no two of $x<y$, $x=y$ and $x>y$ can be true at the same time.\\
    Now let's prove that one of them is always true. \\
    Suppose $x,y \in \Z$ then $x=[(a,b)]$ and $y=[(c,d)]$ also it must be true that
    $a+d<b+c$ or $a+d=b+c$ or $a+d>b+c$ because of Trichotomy of $\N$ if $a+d<b+c$ is
    true then that means that $[(a,b)]<[(c,d)]$, if $a+d=b+c$ then $[(a,b)]=[(c,d)]$ or
    if $a+d>b+c$ then $[(a,b)]>[(c,d)]$.
    \item [(11)]
    Let $x=[(a,b)]$, $y=[(c,d)]$ and $z=[(e,f)]$ then if $x<y$ and $y<z$ that means 
    that $a+d<b+c$ and from the second inequality we have that $c+f<d+e$ then
    by definition $b+c = (a+d) + p$ and $d+e = (c+f)+q$ where $p,q \in \N$ then summing
    both equations we have that
    \begin{align*}
        (b+c)+(d+e) &= (a+d)+p+(c+f)+q \\
        b+e &= (a+f)+(p+q)
    \end{align*}
    If we name $k=p+q$ then by definition $a+f<b+e$ and therefore $[(a,b)]<[(e,f)]$ thus
    $x<z$.
    \item [(13)]
    Let $x=[(a,b)]$, $y=[(c,d)]$ and $z=[(e,f)]$ then if $x<y$ this means that $a+d<b+c$
    and by definition $b+c=(a+d)+p$ where $p\in\N$ also we know that $\hat{0}<z$ so
    $1+f<1+e$ and also by definition this means that $e = f + q$ where $q \in \N$.
    Multiplying both sides of $b+c=(a+d)+p$ with $e$ we get that
    $$e(b+c)=e(a+d)+ep$$
    And doing the same with $f$ we get that
    $$f(a+d)+fp=f(b+c)$$
    Then summing both equations we get that
    $$f(a+d)+e(b+c)+fp=f(b+c)+e(a+d)+ep$$
    Replacing $e=f+q$ on the right hand side of the equation we get that
    $$f(a+d)+e(b+c)+fp=f(b+c)+e(a+d)+fp+qp$$
    and by the Cancellation law we get that
    $$f(a+d)+e(b+c)=f(b+c)+e(a+d)+qp$$
    Which means that $f(b+c)+e(a+d)<f(a+d)+e(b+c)$ then $ae+bf+cf+de<af+be+ce+df$
    and thus $[(ae+bf,af+be)]<[(ce+df,cf+de)]$ therefore $xy<xz$.
    \item [(14)]
    Let's suppose that $\hat{0}=\hat{1}$ we want to arrive to a contradiction, then
    $[(1,1)]=[(1+1,1)]$ so $1+1=1+(1+1)$ and by the Cancellation law we have that
    $1=(1+1)$ which cannot be because there is no $a,b\in\N$ such that $a+b=1$.
    Therefore $\hat{0}\neq\hat{1}$.
    \end{itemize}
    \end{proof}
    \begin{proof}{\textbf{1.3.6}}
    Let us prove the rest of the Theorem 1.3.7 
    \begin{itemize}
    \item [\textbf{1.}] The function $i: \N \rightarrow \Z$ is injective.\\
    Let $i(n)=i(m)$ then by definition of $i$ we have that $[(n+1,1)]=[(m+1,1)]$ thus
    $(n+1)+1=1+(m+1)$ and by the Cancellation law we have that $n=m$.
    \item [\textbf{3.}] $i(1)=\hat{1}$ \\
    By definition of $i$ we have that $i(1)=[(1+1,1)]$ and therefore $i(1)=\hat{1}$.
    \item [\textbf{4b.}] $i(ab)=i(a)i(b)$\\
    By definition of $i$ we have that $i(ab)=[(ab+1,1)]$ then
    \begin{align*}
        i(ab) &= [(ab+1,1)] \\
              &= [(ab+a+b+1+1,a+b+1+1)] \\
              &= [((a+1)(b+1)+1,(a+1)+(b+1))] \\
              &= [(a+1,1)]\cdot [(b+1,1)] \\
              &= i(a)i(b)        
    \end{align*}
    \item [\textbf{4c.}] $a < b$ if and only if $i(a) < i(b)$.\\
    ($\rightarrow$) By definition $a<b$ means that $b=a+p$ where $p\in\N$ then applying
    the function $i$ to both sides of the equation we have that $i(b)=i(a+p)$ and
    because of what we have proven in \textbf{4a} we have that $i(b)=i(a)+i(p)$ which by
    definition means $i(a)< i(b)$.\\
    ($\leftarrow$) By definition $i(a)<i(b)$ means that $[(a+1,1)]<[(b+1,1)]$ and thus
    $(a+1)+1<1+(b+1)$ and because of the Cancellation law we have that $a<b$ as we
    wanted.
    \end{itemize}
    \end{proof}
    \begin{proof}{\textbf{1.3.7}}
    \begin{itemize}
        \item [(1)] ($\rightarrow$)
        Let $x < y$ then by the Addition Law for Order we have that $x + (-x) < y+(-x)$
        then by the Inverses Law for Addition we have that $0 < y+(-x)$ applying the
        Addition Law for Order again we have that $0 +(-y) < (y+(-x))+(-y)$ then by applying
        the Commutative and Associative Law for Addition we have that $-y < (y+(-y))+(-x)$
        and therefore $-y < -x$.\\
        ($\leftarrow$)
        Let $-y <-x$ then by the Addition Law for Order we have that $x + (-y) < x+(-x)$
        then by the Inverses Law for Addition we have that $x+(-y) < 0$ applying the
        Addition Law for Order again we have that $(x + (-y))+y < 0 + y$ then by applying
        the Commutative and Associative Law for Addition we have that $((-y)+y)+x < y$
        and therefore $x < y$.
        \item [(2)] ($\rightarrow$)
        Let $z<0$ and $x<y$ then $-z > 0$ because Lemma 1.4.5 part(8) and by the
        Multiplication Law for Order we have that $x(-z) < y(-z)$ which by the Lemma
        1.3.8 part 6 we know that $-xz < -yz$ thus $xz > yz$ because what we saw in part
        (1) of this problem.\\
        ($\leftarrow$)
        Let $xz>yz$ where $z <0$ because what we saw in part (1) of this problem this
        means that $-xz < -yz$ and then by the Multiplication Law for Order we have that
        $x<y$ because $-z > 0$.
    \end{itemize}
    \end{proof}
    \begin{proof}{\textbf{1.3.8}}
        From Theorem 1.3.9 we know that if $z \in \Z$ there is no $y \in \Z$ such that
        $z<y<z+1$ then there is no $x$ such that $0<x<1$ so if $x>0$ it must be that
        $x \geq 1$.\\
        If $x<0$ then $-x>0$ and as we saw this means that $-x\geq 1$ and by what we
        proved in problem 1.3.7 given that $-1<0$ then $-1 \cdot -x < -1 \cdot 1$
        therefore $-(1 \cdot (-x)) = -(-x) = x < -1$ where in the equalities we are
        using the fact that $(-x)y = -xy = x(-y)$.  
    \end{proof}
    \begin{proof}{\textbf{1.3.9}}
        \begin{itemize}
            \item [(1)]
            From the part (9) of the Lemma 1.4.5 we know that $0 < 1$ then by the
            Addition Law for Order we have that $0 + 1 < 1 + 1$ by the Identity Law for
            Addtion we have that $1 < 1+1$ and now let us call $2$ the following addition
            $2 = 1+1$ therefore $1 < 2$.
            \item [(2)]
            Suppose $2x = 1$ where $x \in \Z$ then as we proved in part (1) we have that
            $2x < 2 = 2 \cdot 1$ then by the Multiplication Law for Order we have that
            $x < 1$ then $x \leq 0$ if $x = 0$ then $2 \cdot 0 = 0 \neq 1$ so must
            happen that $x < 0$ then by Lemma 1.4.5 part (11) $2 \cdot x < 0$ but
            $1 > 0$ which is a contradiction. Therefore $2x \neq 1$. 
        \end{itemize}
    \end{proof}
    \begin{proof}{\textbf{1.3.10}}
        Let's define $G' = i^{-1}(G)$ which is the inverse image of $G$ then
        $G' \subseteq \N$ and because of the Well-Ordering Principle for $\N$ there is
        $m \in G'$ such that $m \leq g$ for all $g \in G'$. By appying $i$ to both sides
        of the inequality we have that $i(m) \leq i(g)$ which we can do because of part
        4c of Theorem 1.3.7.
        We know that $G \subseteq \{x \in \Z ~|~x>\hat{0}\} = i(\N)$ then the elements
        of $G$ have the form $i(a)$ where $a \in G'$ then $i(m),i(g) \in G$ so we want
        to check that $i(m)$ is the minimum then let's suppose that $i(m) > i(g)$ we
        want to arrive to a contradiction then $[(m+1,1)] > [(g+1,1)]$ and thus
        $m+1+1>g+1+1$ by the Cancellation law we have that $m>g$ which is a
        contradiction as we wanted because $m$ is the minimum of $G'$. Therefore
        $i(m) \leq i(g)$ for all $i(g) \in G$.         
    \end{proof}
    \begin{proof}{\textbf{1.3.11}}
        \begin{itemize}
        \item [(1)]
        By adding to both sides of the equation $(-z)$ we get that $(x + z) + (-z) = (y + z) + (-z)$
        and by the Associative Law for Addition we have that $x + (z + (-z)) = y + (z + (-z))$
        then because of the Inverse Law for Addition we have that $x + 0 = y + 0$
        which means that $x = y$ because of the Identity Law for Addition.
        \item [(3)]
        By the Inverses Law for Addition we know that $(x+y) + (-(x+y)) = 0$ then adding
        to both sides of the equation $-x$ and $-y$ we get that
        $(-x) + (-y) + (x+y) + (-(x+y)) = (-x) + (-y)$ then by using multiple times the
        Commutative and Associative Law for addition we get that
        $(x + (-x)) + (y + (-y)) + (-(x+y)) = (-x) + (-y)$ then
        $0 + 0 + (-(x+y)) = (-x) + (-y)$ because of the Inverses Law for Addition and
        therefore $-(x+y) = (-x) + (-y)$ because of the identity law for Addition.
        \item [(4)]
        Let us suppose that $x\cdot 0 \neq 0$ we want to arrive to a contradiction 
        then by the Trychotomy Law it must hold
        that $x \cdot 0 > 0$ or $x \cdot 0 < 0$. Let's suppose that $x \cdot 0 > 0$
        holds then by the Addition Law for Order we have that $(x \cdot 0) + x > 0 + x$
        and because the Identity Law for Multiplication we have that  
        $(x \cdot 0) + (x \cdot 1) > 0 + x$ then by the Identity Law for Addition and
        the Distributive Law we have that $x \cdot (0 + 1) > x$ again by the Identity
        Law for Addition we have that $x \cdot 1 > x$ then $x > x$ because of the
        Identity Law for Multiplication but this cannot be then it must hold that
        $x \cdot 0 < 0$ but by the same type of arguments we see that this cannot be
        either. Therefore it must be that $x \cdot 0 = 0$.
        \item [(5)]
        Suppose that $y = x + k$ where $k \in \Z$ then we get that $xz = yz = (x + k)z$
        and by adding to both sides of the equation $-xz$ we get that
        $xz + (-xz) = (xz + kz) + (-xz)$ then by applying the Associative and 
        Commutative Law for Addition we get that $xz + (-xz) = (xz + (-xz)) + kz$
        and by the Inveses Law for Addition we get that $0 = 0 + kz = kz$ then either
        $k = 0$ or $z = 0$ by the Non Zero Divisors Law, but we know that $z \neq 0$
        then it must be that $k = 0$ therefore $y = x + 0 = x$.
        \item [(7)]
        ($\rightarrow$) If $xy=1$ we have a few cases we need to address to prove that
        either $x=y=1$ or $x=y=-1$. Given that if $x >0$ and $y >0$ then $xy > 0$ and
        if $x>0$ and $y<0$ then $xy < 0$ we can rule a lot of cases by taking into
        account that either both $x$ and $y$ are positive or both are negative. Also
        the case where one of them or both are $0$ is also ruled out because of part (4)
        of this Lemma.
        \begin{itemize}
            \item Let's check first the case where both $x$ and $y$ are positive.
            Let then $x > 1$ and $y > 1$ by multiplying the $y$ inequality by $x$
            we have that $xy>x\cdot 1$ then because of the Identity Law for
            Multiplication we have that $xy > x > 1$ thus $xy \neq 1$.
            \item Now let $x<-1$ and $y<-1$ then $-x > 1 > 0$ because of part (8) of this
            Lemma so we multiply both sides of the $y$ inequality by $-x$ as
            $-xy < -1 \cdot 1$ and because of the Identity Law for Multiplication we
            have that $-xy < -1 < 0$ and then again by the part (8) of this Lemma we have
            that $xy > 1$ thus $xy \neq 1$.
            \item Finally, the only option left is that $x=y=1$ or $x=y=-1$. In the first
            case by the Identity Law for Multiplication we have that
            $xy = 1 \cdot 1 = 1$ and in the second case if $x=y=-1$ we have that
            $xy= (-1) \cdot (-1) = -((-1)\cdot 1)$ because of part (6) of this Lemma and
            then $xy = -(-1) = 1$ because the Identity Law for Multiplication and part
            (2) of this Lemma.            
        \end{itemize}
        Therefore, if $xy=1$ then $x=y=1$ or $x=y=-1$.\\ 
        ($\leftarrow$) As shown before if $x=y=1$ or $x=y=-1$ then $xy=1$.
        \item [(8)]
        ($\rightarrow$) If $x > 0$ then by adding to both sides of the equation $-x$ we
        have that $x+ (-x) > 0 + (-x)$ then because of the Identity Law for Addition we
        have that $x+ (-x) > -x$ and because of the Inverses Law for Addition $0 > -x$.\\
        ($\leftarrow$) If $-x < 0$ then by adding to both sides of the equation $x$ we
        have that $x+ (-x) < x + 0$ and because of the Identity Law and the Inverses Law
        for Addition we have that $0 < x$.\\
        ($\rightarrow$) If $x < 0$ then by adding to both sides of the equation $-x$ we
        have that $x+ (-x) < 0 + (-x)$ and because of the Identity Law and the Inverses
        Law for Addition we have that $0 < -x$.\\
        ($\leftarrow$) If $-x > 0$ then by adding to both sides of the equation $x$ we
        have that $x+ (-x) > x + 0$ and because of the Identity Law and the Inverses
        Law for Addition we have that $0 > x$.
\cleardoublepage
        \item [(10)]
        If $x \leq y$ then either $x = y$ or $x < y$ by definition and in the same way
        if $y \leq x$ then either $y = x$ or $y < x$.\\
        In case $x=y$ and $y=x$ then we are done.\\
        In case $x = y$ and $y < x$ then by replacing $y$ we have that $x<x$ which isn't
        true and thus $x = y$ must be true. The same can be proven for $y=x$ and $x<y$.\\
        In case $x < y$ and $y < x$ then by the Transitive Law $x<x$ which is not true
        and then must be that $x=y$.
        \item [(11)]
        If $x >0$ and $y>0$ then by multipyling the $y$ inequality by $x$ we have that
        $xy > x \cdot 0$ and we can do that because of the Theorem 1.3.5 part (13) and
        because of the result we proved in part (4) of this Lemma then $xy>0$.\\
        If $x >0$ and $y<0$ then by multipyling the $y$ inequality by $x$ we have that
        $xy < x \cdot 0$ and we can do that because of the Theorem 1.3.5 part (13) and
        because of the result we proved in part (4) of this Lemma then $xy < 0$.
    \end{itemize}
    \end{proof}
    \begin{proof}{\textbf{1.4.2}}
        Let $n \in \N$ also we know that $\N$ is defined as $\N = \{x \in \Z ~|~ x > 0\}$ then
        $n \in \Z$ and $n > 0$ by adding $1 > 0$ to both sides of the inequality we have
        that $n +1 > 0 +1 = 1$ where $0 +1 = 1$ because of the Identity Law for Addition
        and as we saw $1>0$ then $n+1>0$ and also $n+1 \in \Z$ therefore $n+1 \in \N$.  
    \end{proof}
    \begin{proof}{\textbf{1.4.3}}\\
        ($\rightarrow$) If $x,y\in\Z$ and $x \leq y$ then by definition either $x=y$ or
        $x<y$ if the last one holds then by adding $-x$ to both sides of the inequality
        we have that $x+(-x)<y+(-x)$ and because of the Inverses Law for Addition we
        have that $0 < y+(-x)$ and if we now add $-y$ to both sides of the inequality
        we have that $(-y)+0<(-y)+(y+(-x))$ and because of the Identity Law for Addition
        and the Associative Law we have that $-y < ((-y)+y)+(-x)$ then again by the
        Inverses Law for Addition we have that $-y < -x$.\\
        But if $x=y$ holds then applying the exact same steps as before we have that
        $-y=-x$.\\
        ($\leftarrow$) If $x,y\in\Z$ and $-y \leq -x$ then by definition either $-y=-x$ or
        $-y<-x$ if the last one holds then by adding $x$ to both sides of the inequality
        we have that $x+(-y)<x+(-x)$ and because of the Inverses Law for Addition we
        have that $x+(-y)< 0$ and if we now add $y$ to both sides of the inequality
        we have that $(x+(-y))+y<0+y$ and because of the Identity Law for Addition
        and the Associative Law we have that $x + ((-y)+y) < y$ then again by the
        Inverses Law for Addition we have that $x < y$.\\
        But if $-y=-x$ holds then applying the exact same steps as before we have that
        $x=y$.
    \end{proof}
\cleardoublepage
    \begin{proof}{\textbf{1.4.4}}
        We defined $\N = \{x \in \Z~|~x>0\}$ and we know because of Theorem 1.4.6 that
        if $z \in \Z$ then there is no $y\in\Z$ such that $z<y<z+1$ then there is no
        $x \in \Z$ such that $0<x<1$ then it must be that $\N = \{x \in \Z~|~x\leq 1\}$.
    \end{proof}
    \begin{proof}{\textbf{1.4.5}}
        If $a<b$ then by adding $1$ to both sides of the equation we have that $a+1<b+1$
        since we saw that there is no $y\in\Z$ such that $b<y<b+1$ then $a+1=b$ or
        $a+1<b$ therefore $a+1\leq b$.
    \end{proof}
    \begin{proof}{\textbf{1.4.6}}
        From problem 1.4.4 we know that $\N = \{x\in\Z~|~x\geq 1\}$ so if $n\in\N$ and
        $n\neq 1$ then it must be that $n > 1$ so by definition there is $b \in \N$ such
        that $n = b + 1$.
    \end{proof}
    \begin{proof}{\textbf{1.4.8}}
        \begin{itemize}
        \item [(1)]
        Let us write $F = \{x\in G ~|~ x+(-a)+1\}$ we want to show by induction that
        $F = \N$ which would be the same as showing that
        $G = \{x\in \Z ~|~ x+(-a)+1 \geq 1\}$. Since $a\in G$ then $a+(-a)+1=1 \in F$.
        Now if $g \in G$ then $g+(-a)+1 \in F$ and by definition we know that $g+1 \in G$
        then $g+1+(-a)+1 = (g+(-a)+1)+1 \in F$ by using the Associative and Commutative
        Law for Addition. Therefore $F=\N$.
        \item [(2)]
        Let us write $F = \{x\in H ~|~ a+(-x)+1\}$ we want to show by induction that
        $F = \N$ which would be the same as showing that
        $H = \{x\in \Z ~|~ 1 \leq a+(-x)+1\}$. Since $a \in H$ then $a+(-a)+1=1 \in F$.
        Now if $h \in H$ then $a+ (-h)+1 \in F$ and by definition we know that
        $h+(-1) \in G$ then $a+(-(h+(-1)))+1 = (a+(-h)+1)+1 \in F$ by using the
        Associative and Comutative Law for Addition and the fact that $-(-1) = 1$.
        Therefore $F = \N$.
        \end{itemize}
    \end{proof}
    \begin{proof}{\textbf{1.5.1}}
        We want to prove that $\asymp$ is an equivalence relation so we must prove that
        it is reflexive, symmetric and transitive. Let $(a,b),(c,d) \in \Z \times \Z^*$.
        We note that $ab=ba$ because of the Commutative Law for Multiplication then
        $(a,b)\asymp(a,b)$ thus $\asymp$ is reflexive. Now suppose that $(a,b)\asymp(c,d)$
        then $ad=bc$ and because of the Commutative Law for Multiplication we have that
        $cb=da$ then $(c,d)\asymp(a,b)$, therefore $\asymp$ is also symmetric.\\
        We also proved that $\asymp$ is transitive. Therefore $\asymp$ is an equivalence
        relation on $\Z \times \Z^*$. 
    \end{proof}
\cleardoublepage
    \begin{proof}{\textbf{1.5.2}}
        Let's prove that $+$ is well-defined for $\Q$.\\
        Let $(x,y),(z,w),(a,b),(c,d) \in \Z \times \Z^*$ and suppose that $[(x,y)] = [(a,b)]$
        and $[(z,w)]=[(c,d)]$ then $(x,y)\asymp(a,b)$ and $(z,w)\asymp(c,d)$ so $xb=ya$
        and $zd=wc$ now multiplying both sides of the the first equation by $dw$ we have
        that $xbdw=yadw$ also we now multiply both sides of the second equation by $yb$
        to obtain $zdyb=wcyb$, now we add both equations
        $$xbdw+zdyb=yadw+wcyb$$
        by the Commutative Law and the Distribute Law we obtain
        $$(xw+yz)(bd)=(yw)(ad+cb)$$
        this means that $[(xw+yz,yw)]=[(ad+cb, bd)]$ therefore $+$ is well-defined.\\
        Let's now prove that the unary operation $^{-1}$ is well-defined for $\Q$.\\
        From $xb=ya$ we deduce that $bx=ay$ then $[(b,a)]=[(y,x)]$ so
        $[(a,b)]^{-1}=[(b,a)]=[(y,x)]=[(x,y)]^{-1}$ therefore $^{-1}$ is well-defined.\\
        Finally, let us prove that $<$ is well-defined for $\Q$.\\
        % We want to get to ad<bc
        Let $[(x,y)]=[(a,b)]$ and $[(z,w)]=[(c,d)]$ then we have that $xb=ya$ and $zd=wc$
        also we have that $[(x,y)]<[(z,w)]$ if $y>0$ and $w>0$ or $y<0$ and $w<0$ then
        $xw<yz$. If $bd > 0$ because $b$ and $d$ are either both positive or
        both negatives then we can multiply the inequality by $bd$ to obtain
        $$xwbd<yzbd$$
        which by the Commutative and Associative Law we have that
        $$(xb)(wd)<(zd)(yb)$$
        then by replacing the values of $xb$ and $zd$ we get that
        $$(ya)(wd)<(wc)(yb)$$
        and again by the Commutative and Associative Law we have that
        $$(yw)(ad)<(yw)(bc)$$
        and since $yw>0$ then $ad<bc$.\\
        If $bd<0$ because either $b<0$ and $d>0$ or $b>0$ and $d<0$ then $-(bd)>0$.
        By multiplying both sides of the inequality with this number we get that 
        $$-(xw)(bd)<-(yz)(bd)$$
        which by the Commutative and Associative Law we have that
        $$-(xb)(wd)<-(zd)(yb)$$
        then by replacing the values of $xb$ and $zd$ we get that
        $$-(ya)(wd)<-(wc)(yb)$$
        and again by the Commutative and Associative Law we have that
        $$-(yw)(ad)<-(yw)(bc)$$
        and since $yw>0$ then $-ad<-bc$ and thus $ad>bc$.\\\\
        Now if either $y<0$ and $w>0$ or $y>0$ and $w<0$ we have that $xw>yz$.
        If $bd < 0$ because either $b<0$ and $d>0$ or $b>0$ and $d<0$ then $-(bd)>0$.
        By multiplying both sides of the inequality with this number we get that 
        $$-(bd)(xw)>-(bd)(yz)$$
        which by the Commutative and Associative Law we have that
        $$-(xb)(wd)>-(zd)(yb)$$
        then by replacing the values of $xb$ and $zd$ we get that
        $$-(ya)(wd)>-(wc)(yb)$$
        and again by the Commutative and Associative Law we have that
        $$-(yw)(ad)>-(yw)(bc)$$
        and since $yw<0$ then $-(yw)>0$ and thus $ad>bc$.\\
        Finally, if $bd > 0$ because $b$ and $d$ are either both positive or
        both negatives then we can multiply the inequality by $bd$ to obtain
        $$xwbd>yzbd$$
        which by the Commutative and Associative Law we have that
        $$(xb)(wd)>(zd)(yb)$$
        then by replacing the values of $xb$ and $zd$ we get that
        $$(ya)(wd)>(wc)(yb)$$
        and again by the Commutative and Associative Law we have that
        $$(yw)(ad)>(yw)(bc)$$
        and since $yw<0$ then $-(yw)>0$ so we have that 
        $$-(yw)(ad)<-(yw)(bc)$$
        and thus $-ad<-bc$ which means that $ad>bc$.\\
        Therefore the $<$ operation is well-defined.
    \end{proof}
\cleardoublepage
    \begin{proof}{\textbf{1.5.3}}
        \begin{itemize}
        \item [(1)]
        ($\rightarrow$) If $[(x,y)]=\bar{0}=[(0,1)]$ then by definition $x\cdot 1 = y \cdot 0$
        and because of the Identity Law for Multiplication and the part (4) of the Lemma
        1.4.5 we have that $x=0$.\\
        ($\leftarrow$) If $x = 0$ then because of the Identity Law for Multiplication
        and the part (4) of the Lemma 1.4.5 we can write that $x\cdot 1=y\cdot 0$ where
        $y\in\Z^*$ then $[(x,y)]=[(0,1)]=\bar{0}$.
        \item [(2)]
        ($\rightarrow$) If $[(x,y)]=\bar{1}=[(1,1)]$ then by definition $x\cdot 1 = y\cdot 1$ and
        because of the Identity Law for Multiplication we have that $x=y$.\\
        ($\leftarrow$) If $x = y$ then because of the Identity Law for Multiplication
        we can write that $x\cdot 1=y\cdot 1$ and by definition that means $[(x,y)]=[(1,1)]=\bar{1}$.
        \item [(3)]
        ($\rightarrow$) If $\bar{0}=[(0,1)]<[(x,y)]$ then by definition if $y>0$ this is
        $0\cdot y < 1 \cdot x$ then because of the Identity Law for Multiplication and
        the part (4) of the Lemma 1.4.5 we have that $x>0$ and therefore $xy>0$ because
        both are positive numbers.\\
        If $y<0$ then by definition $[(0,1)]<[(x,y)]$ means that $0\cdot y>1\cdot x$
        then because of the Identity Law for Multiplication and the part (4) of the
        Lemma 1.4.5 we have that $x<0$ and therefore $xy>0$ because both are negative
        numbers.\\
        ($\leftarrow$) If $xy>0$ then either $x>0$ and $y>0$ or $x<0$ and $y<0$. In the
        first case i.e. $x>0$ and $y>0$ we have can write the first inequality as
        $0\cdot y<1 \cdot x$ because of the Identity Law for Multiplication and the part
        (4) of the Lemma 1.4.5 which means that $[(0,1)]<[(x,y)]$.\\
        If $x<0$ and $y<0$ again from the first inequality we can write that
        $0\cdot y>1\cdot x$ because of the Identity Law for Multiplication and the part
        (4) of the Lemma 1.4.5 which means that $[(0,1)]<[(x,y)]$.
        \end{itemize}
    \end{proof}
    \begin{proof}{\textbf{1.5.4}}
        \begin{itemize}
        \item [(1)]
        From the definition of the $+$ operation we have that
        \begin{align*}
            (r+s)+t &= ([(x,y)]+[(z,w)]) + [(u,v)]\\
                    &= [(xw+yz,yw)]+ [(u,v)]\\
                    &= [((xw+yz)v+(yw)u,(yw)v)]
        \end{align*}
        Because of the Distributive, Commutative and Associative Law for Addition and
        Multiplication of $\Z$ we have that
        \begin{align*}
            [((xw+yz)v+(yw)u,(yw)v)] &= [(x(wv)+y(zv+wu),y(wv))]\\
                                     &= [(x, y)]+[(zv+wu, wv)]\\
                                     &= [(x, y)] + ([(z, w)]+[(u, v)])\\
                                     &= r + (s + t)     
        \end{align*}
        \item [(2)]
        From the definition of the $+$ operation we have that
        \begin{align*}
            r+s &= [(x,y)]+[(z,w)]\\
                &= [(xw+yz,yw)]
        \end{align*}
        Because of the Commutative law for Multiplication and Addition of $\Z$ we have
        that
        $$[(xw+yz,yw)] = [(zy+wx,wy)]$$
        then
        $$[(zy+wx,wy)] = [(z,w)] + [(x,y)] = s + r$$
        \item [(3)]
        From the definition of the $+$ operation and the $\bar{0}$ element we have that
        \begin{align*}
            r+\bar{0} &= [(x,y)]+[(0,1)]\\
                &= [(x\cdot 1+y\cdot 0,y\cdot 1)] \\
                &= [(x,y)] = r
        \end{align*}
        Where we are using the Identity Law for Multiplication and that $z\cdot 0 = 0$
        where $z \in \Z$.
        \item [(5)]
        From the definition of the $\cdot$ operation we have that
        \begin{align*}
            (rs)t &= ([(x,y)]\cdot [(z,w)]) \cdot [(u,v)] \\
                &= [(xz,yw)]\cdot [(u,v)] \\
                &= [((xz)u, (yw)v)] \\
                &= [(x(zu), y(wv))] \quad \begin{aligned}[t]&\text{Because of the Associative and} \\
                                                            &\text{Commutative Law for Multiplication}\end{aligned}\\
                &= [(x, y)]\cdot[(zu, wv)]\\
                &= [(x, y)]\cdot([(z, w)]\cdot[(u,v)]) = r(st)
        \end{align*}
        \item [(6)]
        From the definition of the $\cdot$ operation we have that
        \begin{align*}
            rs &= [(x,y)]\cdot [(z,w)] \\
               &= [(xz,yw)]\\
               &= [(zx,wy)] \quad \text{Because of the Commutative Law}\\
               &= [(z,w)]\cdot[(x,y)] = sr
        \end{align*}
        \item [(8)]
        From the definition of $r^{-1} = [(y,x)]$ we have that
        \begin{align*}
            rr^{-1} &= [(x,y)] \cdot [(y,x)]\\
                &= [(xy,yx)]\\
                &= [(yx,yx)] \quad \text{because of the Commutative Law}\\
                &= \bar{1} \quad\quad \text{because of Problem 1.5.3 part (2)}
        \end{align*}
        \item [(9)]
        From the definition of $+$ and $\cdot$ operations we have that
        \begin{align*}
            rs +rt &= [(x,y)]\cdot[(z,w)] + [(x,y)]\cdot[u,v]\\
                &= [(xz,yw)] + [(xu,yv)]\\
                &= [((xz)(yv) + (yw)(xu),(yw)(yv))]\\
                &= [(y(xzv + wxu), y(ywv))]\quad \begin{aligned} &\text{because of the Distributive}\\
                                                                 &\text{and Associative Law}\end{aligned}\\
                &= [(y,y)]\cdot[(xzv + wxu, y(wv))]\\
                &= \bar{1} \cdot[(xzv + wxu, y(wv))] \quad \text{because of Problem 1.5.3 part (2)}\\
                &= [(xzv + wxu, y(wv))] \quad \text{because of the Identity Law}\\
                &= [(x(zv+wu),y(wv))] \quad \begin{aligned} &\text{because of the Distributive}\\
                                                            &\text{and Associative Law}\end{aligned}\\
                &= [(x,y)]\cdot[(zv+wu,wv)]\\
                &= [(x,y)]\cdot([(z,w)]+[(u,v)]) = r(s+t)
        \end{align*}
        \item [(11)]
        We know that if $x<y$ if and only if $y-x \in P$ so given that $r<s$ and $s<t$
        then $s-r, t-s \in P$ let us name $s-r = u$ and $t-s = v$ then
        $u+v = (s-r) + (t-s) = t-r$ and we know that if $x,y \in P$ then $x+y \in P$
        then $u+v = t-r \in P$ and therefore $r<t$.
        \item [(12)]
        Since $r<s$ then we know that $s-r \in P$ now if we compute $(s+t)-(r+t)=s-r$
        we have that $(s+t)-(r+t) \in P$ and therefore $r+t<s+t$.
        \item [(14)]
        Let us suppose that $\bar{0} = \bar{1}$ we want to arrive to a contradiction
        then $[(0,1)] = [(1,1)]$ which means that $0\cdot 1 = 1 \cdot 1$ but this means
        that $0 = 1$ which is a contradiction because what we proved in Theorem 1.3.5
        part (14). Therefore $\bar{0} \neq \bar{1}$.
        \end{itemize}
    \end{proof}
    \begin{proof}{\textbf{1.5.5}}
    \begin{itemize}
        \item [(1)]
        We want to prove that the function $i:\Z\rightarrow\Q$ is injective, then we
        proceed as follows, let $x_1, x_2 \in \Z$ then if $i(x_1)=i(x_2)$ we have that
        $[(x_1, 1)] = [(x_2, 1)]$ so $x_1\cdot 1=1 \cdot x_2$ and therefore $x_1 = x_2$
        and the function is injective.
        \item [(2)]
        By definition $i(0) = [(0,1)] =\bar{0}$ and $i(1)=[(1,1)]=\bar{1}$
        \item [(3)]
        \begin{itemize}
            \item [(a)]
            By definition of $i$ function and the $+$ operation we have that 
            \begin{align*}
                i(x)+i(y) &= [(x,1)]+[(y,1)]\\
                    &= [(x\cdot 1+1\cdot y, 1\cdot 1)]\\
                    &= [(x+y, 1)] \quad\begin{aligned}\text{because of the Identity}\\
                        \text{Law for Multiplication}\end{aligned} \\
                    &= i(x+y)
            \end{align*}
            \item [(b)]
            By definition of the $i$ function we have that $i(-x) = [(-x,1)]$ and because
            of the defintion of the $-$ unary operation we have that $[(-x,1)]=-[(x,1)]$
            therefore because of the definition of the function $i$ again we get that
            $-[(x,1)]=-i(x)$.
            \item [(c)]
            By definition of $i$ function and the $\cdot$ operation we have that
            \begin{align*}
                i(x)i(y) &= [(x,1)]\cdot[(y,1)]\\
                    &= [(xy,1\cdot 1)]\\
                    &= [(xy,1)]\quad \text{Because of the Identity Law}\\
                    &= i(xy)
            \end{align*}
            \item [(d)]
            ($\rightarrow$) If $x<y$ then because of the Identity Law for Multiplication
            we have that $x\cdot 1<1\cdot y$ and because of the definition of the $<$
            operation we can write that $[(x,1)]<[(y,1)]$ which means that $i(x)<i(y)$.\\
            ($\leftarrow$) If $i(x)<i(y)$ then by the definition of the $i$ function we
            have that $[(x,1)]<[(y,1)]$. Because of the definition of the $<$ operation 
            and given that $1>0$ as we proved earlier in the Lemma 1.4.5 we have that
            $x\cdot 1<1\cdot y$ and because of the Identity Law for Multiplication we
            have that $x<y$.
        \end{itemize}
    \end{itemize}
    \end{proof}
    \begin{proof}{\textbf{1.5.6}}
    \begin{itemize}
        \item [(2)]
        If $r<s$ then because of the Addition Law for Order we have that $r+(-r)<s+(-r)$
        which means because of the Inverses Law for Addition that $0<s+(-r)$ then again
        by the Addition Law for Order we have that $(-s)+0<((-s)+s)+(-r)$ then it follows
        because of the Identity Law for Addition and the Inverses Law for Addition that
        $-s<-r$ as we wanted.
        \item [(3)]
        Let us suppose that $r\cdot 0 \neq 0$ we want to arrive to a contradiction 
        then by the Trychotomy Law it must hold
        that $r \cdot 0 > 0$ or $r \cdot 0 < 0$. Let's suppose that $r \cdot 0 > 0$
        holds then by the Addition Law for Order we have that $(r \cdot 0) + r > 0 + r$
        and because the Identity Law for Multiplication we have that  
        $(r \cdot 0) + (r \cdot 1) > 0 + r$ then by the Identity Law for Addition and
        the Distributive Law we have that $r \cdot (0 + 1) > r$ again by the Identity
        Law for Addition we have that $r \cdot 1 > r$ then $r > r$ because of the
        Identity Law for Multiplication but this cannot be then it must hold that
        $r \cdot 0 < 0$ but by the same type of arguments we see that this cannot be
        either. Therefore it must be that $r \cdot 0 = 0$.
\cleardoublepage
        \item [(1)]
        We know from the Theorem 1.5.5 (14) that $0\neq 1$ and by the Trychotomy Law
        then either $0<1$ or $1<0$. Let us suppose that $1<0$ and because of part (2)
        of this problem we have that $-0=0<-1$ then we multiply both sides of the 
        inequality by $-1$ as $1\cdot (-1) < 0\cdot(-1)$ because what we proved in
        part (3) we have that $1\cdot(-1)<0$ and finally because of the Identity Law for
        Multiplication we have that $-1<0$ which is a contradiction to what we showed
        earlier. Therefore it must be that $0<1$.\\
        On the other hand, because of what we proved in part (2) of this problem and
        starting from the fact that $0<1$ we have that $-1<-0=0$
        as we wanted. It follows then that $-1<0<1$.
        \item [(4)]
        If $r>0$ and $s>0$ then
        \begin{align*}
            r+s &> 0+s \quad \text{by the Addition Law for Order}\\
                &= s \quad \text{by the Identity Law for Addition}\\
                &> 0 \quad \text{by the Transitive Law}
        \end{align*}
        Also, using the $r>0$ inequality we have that
        \begin{align*}
            rs &> 0\cdot s \quad \text{by Multiplication Law for Order since }s>0\\
                &= 0 \quad \text{because Problem (3)}
        \end{align*}
        Therefore $r+s>0$ and $rs>0$ as we wanted.
        \item [(5)]
        We know that $rr^{-1}=1$ then $rr^{-1}>0$ because $1>0$ as we proved. It followss
        then that $r^{-1} \neq 0$ because what we proved in problem (3). Let us suppose
        now that $r^{-1}<0$ then since $r>0$ we can multiply both sides of the inequality
        to obtain that $rr^{-1}<r \cdot 0 = 0$ where we used the result of problem (3)
        and we have a contradiction to the fact that $rr^{-1}>0$. Therefore must be that
        $r^{-1}>0$.
        \item [(6)]
        Because of the Identity Law for Multiplication we can write $r<s$ as
        $1\cdot r<s\cdot 1$. Because of the Transitive Law since $r>0$ we have that $s>0$ too.
        If follows then because of Lemma 1.5.8 (6) that $\frac{1}{s}<\frac{1}{r}$ as we
        wanted.
        \item [(7)]
        Given that $s>0$ we can multiply both sides of the inequality $r<p$ by $s$ to get
        $rs<ps$ also because of the Transitive Law $p>0$ so we can multiply the inequality
        $s<q$ by $p$ to get $ps<pq$ and therefore by the Transitive Law we have that
        $rs<pq$ as we wanted. 
    \end{itemize}
    \end{proof}
    \begin{proof}{\textbf{1.5.7}}
    \begin{itemize}
        \item [(1)]
        We know that $0<1$ from problem 1.5.6 (1) and because of the Addition Law for
        Order we can add to both sides $1$ to get $0+1<1+1$. Let us call $2$ the addition 
        $1+1$ therefore applying this definition and by the Identity Law for Addition we
        have that $1<2$.
        \item [(2)]
        Since $s,t \in \Q$ then we can write them as $s=\frac{a}{b}$ and $t=\frac{c}{d}$
        where $a,b,c,d \in \Z$.\\
        Let us compute $\frac{s + t}{2} = (s+t)\cdot \frac{1}{2}$ as
        \begin{align*}
            (s+t)\cdot \frac{1}{2} &= (\frac{a}{b} + \frac{c}{d})\cdot \frac{1}{2}\\
                &= (\frac{ad+bc}{bd})\cdot \frac{1}{2} \quad \text{because Lemma 1.5.8 (2)}\\
                &= \frac{ad+bc}{2 \cdot (bd)} \quad \text{because Lemma 1.5.8 (4)}
        \end{align*}
        Therefore since $ad+bc \in \Z$ and $2\cdot (bd) \in \Z$ and $\frac{s+t}{2}$ can
        be written as a fraction then $\frac{s+t}{2} \in \Q$.\\
        Let us now prove that $s< \frac{s+t}{2}<t$ as follows.
        \begin{align*}
            s&<t\\
            s+s&<s+t \quad \text{by the Addition Law for Order}\\
            s(1+1)&<s+t \quad \text{by Distributive and Identity Law}\\
            s\cdot 2 &< s+t \quad \text{by Definition of 2}\\
            s \cdot (2\cdot 2^{-1}) &< (s+t) \cdot 2^{-1}\quad
                \begin{aligned}
                    &\text{by Multiplication Law for Order}\\
                    &\text{since }2>0\text{ then }2^{-1}>0
                \end{aligned}\\
            s &< \frac{s+t}{2} \quad \text{by Inverses Law for Multiplication}
        \end{align*}
        On the other hand, we have that
        \begin{align*}
            s&<t\\
            s+t&<t+t \quad \text{by the Addition Law for Order}\\
            s+t&<t(1+1) \quad \text{by Distributive and Identity Law}\\
            s+t &< t\cdot 2 \quad \text{by Definition of 2}\\
            (s+t) \cdot 2^{-1} &< t \cdot (2\cdot 2^{-1}) \quad
                \begin{aligned}
                    &\text{by Multiplication Law for Order}\\
                    &\text{since }2>0\text{ then }2^{-1}>0
                \end{aligned}\\
            \frac{s+t}{2} &< t \quad \text{by Inverses Law for Multiplication}
        \end{align*}
        Therefore by joining both results we have that $s<\frac{s+t}{2}<t$.
    \end{itemize}        
    \end{proof}
    \begin{proof}{\textbf{1.5.8}}
    \begin{itemize}
        \item [(1)]
        We know that $r=\frac{a}{b}>\frac{0}{1}=0$ where $a,b \in \Z$ and $b \neq 0$.\\
        Let us suppose that $b>0$ then because of the Lemma 1.5.8 (6) we have that
        $a\cdot 1 > b\cdot 0$. It follows then that $a>0$ because of the Identity Law
        for Multiplication and the result of Exercise 1.5.6 (3).\\
        Let us now suppose that $b<0$ then because of the Lemma 1.5.8 (6) we have that
        $a\cdot 1 < b\cdot 0$. It follows then that $a<0$ because of the Identity Law
        for Multiplication and the result of Exercise 1.5.6 (3).
        \item [(2)]
        Since $r\in \Q$ then we can write $r=\frac{a}{b}$. Where $a,b \in \Z$.
        If $a>0$ and $b>0$ then we are done. So let $a<0$ and $b<0$ then $-a>0$ and $-b>0$. 
        We know also that
        $$a\cdot(-b) = (-a)\cdot b$$
        because of Lemma 1.4.5 (6) and by writing
        $$a\cdot(-b) = b\cdot (-a)$$
        then this means that $r = \frac{a}{b} = \frac{-a}{-b}$ because of the Lemma 1.5.8 (1).\\
        Therefore we have found $m=-a$ and $n=-b$ such that $m>0$, $n>0$ and $r = \frac{m}{n}$.
    \end{itemize}
    \end{proof}
    \begin{proof}{\textbf{1.5.9}}
    \begin{itemize}
        \item [(1)]
        Let us define $\N = \{\frac{a}{1}~|~a>0\}$.\\
        We need to find $n>\frac{s}{r}$ where $n\in \N$ and since $\frac{s}{r} \in \Q$
        is of the form $\frac{a}{b}$ where $a,b \in \Z$ then we need to find
        $n=\frac{n}{1}>\frac{a}{b}$ but this means because of Lemma 1.5.8 (1) that
        $nb>1\cdot a = a$ and because Exercise 1.5.8 (2) we know there is $b > 0$ and
        $a>0$ such that $\frac{a}{b}>0$ then $a,b \in \N$ and therefore $b \geq 1$. So
        if $b=1$ then we take $n=a+1$ and then $(a+1)b = a+1>a$ which is true, and if
        $b>1$ then by multiplying both sides by $a$ we have that $ab>a$ so it's enough
        to select $n=a$.
        \item [(2)]
        We need to find $m$ such that $\frac{1}{m}<r$ or $m>    \frac{1}{r}$ if $r>0$ and $r\in \Q$ then we can
        write $r$ as $r = \frac{a}{b}$ where there is $a>0$ and $b>0$ because of
        Exercise 1.5.8 (2) and then $a,b \in \N$. It follows then that
        $\frac{1}{r} = \frac{b}{a}$ so in other words we want to find because of
        Lemma 1.5.8 (1) some $m$ such that $ma>b$. Since $a \in \N$ then $a \geq 1$. If
        $a=1$ then by taking $m=b+1$ we are good to go since $m=b+1>b$, and if $a>1$
        then by multiplying both sides by $b$ we have that $ab>b$ so it's enough
        to select $m=b$.
\cleardoublepage
        \item [(3)]
        We want to find some $k \in \N$ such that
        $$\left(r+\frac{1}{k}\right)^2=r^2+2\frac{r}{k}+\frac{1}{k^2}<p$$
        but since $\frac{1}{k^2}<\frac{1}{k}$ the above inequality is going to be
        satisfied if the folllowing one is satisfied too
        $$r^2+2\frac{r}{k}+\frac{1}{k}<p$$
        then
        \begin{align*}
            \frac{2r}{k}+\frac{1}{k} &< p-r^2 \quad \text{by adding to both sides }-r^2\\
            \frac{2r+1}{k} &< p- r^2 \quad \text{by adding both left terms}\\
            2r+1 &< (p-r^2)k \quad \text{by multiplying both sides by }k>0\\
            \frac{2r+1}{p-r^2} &< k \quad \begin{aligned}&\text{by multiplying both sides by }(p-r^2)^{-1}>0\\
                        &\text{since }p>0 \text{, }r>0\text{ and }r^2<p\end{aligned}
        \end{align*}
        Now given that $r,p \in \Q$ we can write them as fractions and because of the
        Exercise 1.5.8 (2) we know there is $a>0$, $b>0$, $c>0$ and $d>0$ i.e
        $a,b,c,d \in \N$ such that $r=\frac{a}{b}$ and $p=\frac{c}{d}$. Also, since
        $p-r^2 \in \Q$ and $p-r^2>0$ we can write it as $p-r^2=\frac{e}{f}$ such that
        $e,f \in \N$. Then we want to find $k$ such that
        \begin{align*}
            \frac{\frac{2a}{b}+1}{\frac{e}{f}} &< k\\
            \frac{\frac{2a+ b}{b}}{\frac{e}{f}} &< k \\
            \frac{f(2a+b)}{eb} &< k \\
            f(2a+b) &< (eb)k \quad \text{because of Lemma 1.5.8 (6)}
        \end{align*}
        Since $a,b,e,f \in \N$ and $2 \in \N$ then $eb \in \N$ and $f(2a+b) \in \N$
        so $eb$ must be $eb \geq 1$. If $eb=1$ then by taking $k = f(2a+b) + 1$ is
        enough given that $f(2a+b) < f(2a+b) + 1 = k$ and if $eb > 1$ then by
        multiplying by $f(2a+b)$ both sides we have that $f(2a+b) < (eb)(f(2a+b))$
        then taking $k = f(2a+b)$ is enough. 
    \end{itemize}
    \end{proof}
\cleardoublepage 
    \begin{proof}{\textbf{1.6.1}}
        Let us prove by contradiction that $B-A$ has an infinite amount of elements.\\
        The set $B-A$ is defined as
        $B-A = \{b \in B ~|~ b<a \text{ for all }a \in A\}$ and let us suppose that 
        $B-A = \{b_1, b_2, b_3, ... , b_n\}$
        where $n \in \N$ let's take $z = min(\{b_1,b_2,b_3,...,b_n\})$ by definition
        $z \in B$ and $z<a$ and since $B$ is a Dedekin cut then there is some $w \in B$
        such that $w<z$ and by the Transitive Law we have that $w<a$, but $z$ was the
        minimum value, therefore this is a contradiction and it follows that $B-A$ has
        an infinite amount of elements.
    \end{proof}
    \begin{proof}{\textbf{1.6.2}}
    \begin{itemize}
    \item [(1)] Let us prove that $T$ is a Dedekin cut.
    \begin{itemize}
        \item [(a)] Given that $T$ is defined as $T =\{x \in \Q~|~x>0 \text{ and }x^2>2\}$
        then $0 \notin T$ but $0 \in \Q$ then $T \neq \Q$. From Exercise 1.5.7 part (1)
        we know that $1<2$ then since $2>0$ we can multiply both sides of the inequality
        by $2$ therefore $2<2\cdot 2 = 2^2$ and $2 \in \Q$ so we see that $2 \in T$ it
        follows then that $T \neq \emptyset$.
        \item [(b)] Let $t \in T$ and $y \in \Q$ and let us suppose that $y > t$. Given
        that $t>0$ we have that $y>0$ by the Transitivity Law. Let us now multiply the
        inequality by $t$ so we have that $yt>t^2$ and let us also multiply by $y$ to
        obtain $y^2>yt$ therefore by the Transitivity Law we have that $2<t^2<y^2$. It
        follows then that $y \in T$.
        \item [(c)] Let $t \in T$ then $t>0$ and $t^2>2$ we want to find some $r$ such
        that $0<r<t$ and $2<r^2<t^2$ so let us take $r = t - \frac{1}{k}$ where
        $k \in \N$. We want that $t-\frac{1}{k}>0$ then $t > \frac{1}{k}$ and we know
        such a $k$ exists because of Problem 1.5.9 part (2).\\
        We also want that $2<(t-\frac{1}{k'})^2$ where $k' \in \N$ might not be equal to
        $k$ then we want some $k'$ such that  $2 < t^2 - \frac{2t}{k'} +\frac{1}{k'^2}$
        but this means that finding a $k'$ such that $2 < t^2 - \frac{2t}{k'}$ is also
        good to go. It follows that 
        \begin{align*}
            2 - t^2 &< -\frac{2t}{k'} \\  
            \frac{2 - t^2}{2t} &< -\frac{1}{k'} \quad \text{because }t >0\\
            \frac{t^2 - 2}{2t} &> \frac{1}{k'} \quad \text{multiplying by }-1
        \end{align*}
        Given that $t \in \Q$ we notice that $\frac{t^2- 2}{2t} \in \Q$ and since
        $t^2>2$ then $\frac{t^2- 2}{2t} > 0$. It follows then that because of the
        Problem 1.5.9 part (2) we know that such $k' \in \N$ exist.
        Therefore if we take $k'' = max(k, k')$ then $t - \frac{1}{k''} >0$ and
        $2<(t -\frac{1}{k''})^2$ then $t -\frac{1}{k''} \in T$.
    \end{itemize}
\cleardoublepage
    \item [(2)]
    Let $y \in D_r$ then $y>r$ for some $r \in \Q$, clearly $r>0$ hence $y^2>r^2$ but
    also $T = D_r$ so $y \in T$ it follows that $y>0$ and $y^2>2$.\\
    If $y>0$ and $y^2 >2$ then $y \in T$ and so $y \in D_r$ so $y>r$. So we see that
    $y>r$ implies that $y^2>2$ and $y^2>2$ implies that $y>r$.\\
    Let us now assume that $r^2>2$ if we take $y=r$ then $r^2=y^2>2$ and from what we 
    saw this implies that $r=y>r$ which is a contradiction.
    Now let $r^2<2$ then we know there is some $q \in \Q$ such that $r^2<q^2<2$  but
    this means that $r<q$ which implies by what we saw that $q^2>2$ which is a
    contradiction. Therefore must be that $r^2 = 2$. 
    \end{itemize}
    \end{proof}
    \begin{proof}{\textbf{1.6.3}}
        We want to prove that
        $$M = \{r \in \Q ~|~ r=ab \text{ for some }a\in A\text{ and }b \in B\}$$
        is a Dedekin cut, where we suppose that $0 \in \Q -A$ and $0 \in \Q -B$, and
        $A$ and $B$ are Dedekin cuts.\\
        So we proceed to prove the three parts of the Dedekin cuts definition.
        \begin{itemize}
        \item [(a)] We know that $A \neq \emptyset$ and $A \neq \Q$, also
        $B \neq \emptyset$ and $B \neq \Q$. Let $x \in A$ and $y \in B$ then $xy \in M$
        so $M \neq \emptyset$.\\
        We also know that $0 \notin A$ and $0 \notin B$ then $0\cdot 0 = 0 \notin M$ but
        $0 \in \Q$ therefore $M \neq \Q$.
        \item [(b)] Let $t \in M$ and $y \in \Q$, and suppose $y \geq t$, we know that
        $t=ab$ for some $a \in A$ and $b \in B$. Then we can write $y=\frac{yb}{b}$ and
        because $y \geq t$ then $\frac{y}{b} \geq a$ so $\frac{y}{b} \in A$ because $A$
        is a Dedekin cut. Therefore $y=(\frac{y}{b})\cdot b \in M$.
        \item [(c)] Let $t \in M$ then $t = ab$ for some $a \in A$ and $b \in B$. Given
        that $0 \in \Q -A$ and $0 \in \Q -B$ then $0 \notin A$ and $0 \notin B$ and
        therefore $a>0$ and $b>0$. It follows from the definition of the Dedekin cuts
        that there is some $p<a$ and $q<b$ such that $p \in A$ and $q \in B$ so
        $pq \in M$ and because $0<p<a$ and $0<q<b$ we have that $pq < ab$. 
        \end{itemize}
        Therefore $M$ is a Dedekin cut.
    \end{proof}
\cleardoublepage
    \begin{proof}{\textbf{1.6.4}}
        \item [(1)] ($\rightarrow$) If $A \subsetneqq D_r = \{x \in \Q ~|~ x>r\}$ then
        there is some $q \in D_r$ such that $q \notin A$ then $q>r$ and $q \in \Q$ by
        definition of $D_r$. It follows that $q \in \Q - A$.\\
        ($\leftarrow$) If $q \in \Q - A$ and $q > r$ then $q \notin A$ but $q \in \Q$ so
        by definition $q \in D_r$. It follows that $q \in D_r - A$ but this means that
        $A \subsetneqq D_r$.
        \item [(2)] Let's prove first that $A \subseteq D_r$ if and only if $r \in \Q -A$.\\
        ($\rightarrow$) Let $A \subseteq D_r$ as we know $r \notin D_r$ so $r \notin A$
        but $r \in \Q$ by definition of $D_r$. It follows that $r \in \Q - A$.\\
        ($\leftarrow$) Let $r \in \Q - A$ then $r \notin A$. Since $A$ is a Dedekin cut
        then there is no $x \in A$ such that $x<r$. So if $x \in A$ then $x>r$ but then
        $x \in D_r$ and since $x$ was arbitrary therefore $A \subseteq D_r$.\\
        Now let us prove that $r \in \Q-A$ if and only if $r<a$ for all $a \in A$.\\
        ($\rightarrow$) If $r \in \Q - A$ then $r \notin A$ and since $A$ is a Dedekin
        cut, all $a \in A$ must be $r<a$ otherwise we have a contradiction to the fact
        that $A$ is a Dedekin cut.
        ($\leftarrow$) If $r<a$ for all $a \in A$ then by definition $r \notin A$ but
        $r \in \Q$ by definition so this means that $r \in \Q-A$.
    \end{proof}
    \begin{proof}{\textbf{1.7.1}}
    \begin{itemize}
        \item [(1)]
        Let $x \in D_{-r}$ then $x \in \Q$ and $-r<x$ so $-x<r$. We know that 
        $-D_r = \{x \in \Q ~|~ -x<c\text{ for some }c \in \Q-D_r\}$. Also given that 
        $\Q-D_r$ is defined as $\Q - D_r = \{y \in \Q ~|~ y<x \text{ for all } x \in D_r\}$
        and by definition of $D_r$, $r < x$ where $x \in D_r$ we have that
        $r \in \Q - D_r$ then $x \in -D_r$. Therefore $D_{-r} \subseteq -D_r$.\\
        Let $x \in -D_{r}$ then $x \in \Q$ and $-x<c$ for some $c \in \Q-D_r$. As we
        saw $r \in \Q-D_r$ and by definition of $D_r$ we have that $x>r$ so $-x<-r<r$
        but then $x>-r$ so $x \in D_{-r}$ and therefore $-D_r \subseteq D_{-r}$.\\
        It follows that $D_{-r} = -D_r$.
        \item [(2)]
        Let $r>0$ and $D_{r^{-1}} = \{x \in \Q ~|~ x>\frac{1}{r}\}$ then $[D_r]^{-1}$
        is defined as
        $[D_r]^{-1} = \{x \in \Q ~|~ x>0 \text{ and }\frac{1}{x}<c\text{ for some }c \in \Q-D_r\}$
        % and if $D_r < D_0$ then $[D_r]^{-1}$ is defined as $[D_r]^{-1} = -(-D_r)^{-1}$.\\
        
        Let $x \in D_{r^{-1}}$ then from the definition we have that $r > \frac{1}{x}$
        which is possible since $x>\frac{1}{r}>0$ and we know that
        $r \in \Q-D_r$ then $x \in [D_r]^{-1}$ which means that $D_{r^{-1}}\subseteq [D_r]^{-1}$.
        
        Let $x \in [D_r]^{-1}$ then there is some $c \in \Q - D_r$ such that
        $\frac{1}{x}<c$ and we know $r \in \Q - D_r$ so if we take $c=r$ then
        $\frac{1}{x}<r$ so $\frac{1}{r}<x$ it follows then that $x \in D_{r^{-1}}$ which
        means that $[D_r]^{-1}\subseteq D_{r^{-1}}$.
        Therefore when $r>0$ we have that $D_{r^{-1}}=[D_r]^{-1}$

        Let now $r<0$ then $D_{r^{-1}} = D_{-(-r)^{-1}}$ and because of part 1 of this
        Problem we have that $D_{-(-r)^{-1}} = -D_{(-r)^{-1}}$ and now because $-r>0$ we
        have that $-D_{(-r)^{-1}} = -[D_{-r}]^{-1} = -[-D_r]^{-1} = [D_r]^{-1}$.        
    \end{itemize}
    \end{proof}
\cleardoublepage
    \begin{proof}{\textbf{1.7.2}}
        \begin{itemize}
            \item [(1)] Given that $A > D_0$ and $B > D_0$ then $AB$ is defined as
            $$AB = \{r \in \Q ~|~ r=ab \text{ for some }a\in A \text{ and }b \in B\}$$
            given that $a>0$ and $b>0$ then $r=ab>0$ so $AB \subsetneqq D_0$ which by
            definition means that $AB > D_0$.
            \item [(2)] Given that $A > D_0$ then $A^{-1}$ is defined as
            $$A^{-1} = \{r \in \Q ~|~ r>0 and \frac{1}{r}<c \text{ for some }c \in \Q-A\}$$ 
            then by definition $r>0$ which means that $A^{-1} \subsetneqq D_0$ it follows
            that $A^{-1} > D_0$.
        \end{itemize}
    \end{proof}
    \begin{proof}{\textbf{1.7.7}}
        \begin{itemize}
            \item [(1)] Let $i(r_1) = i(r_2)$ then $D_{r_1} = D_{r_2}$ let
            $x \in D_{r_1}$ then $x \in D_{r_2}$ so $x>r_1$ and $x>r_2$ then if
            $r_1>r_2$ if follows that exists a $r_1>\frac{r_1+r_2}{2}>r_2$ but
            $\frac{r_1+r_2}{2} \in D_{r_2}$ and $\frac{r_1+r_2}{2} \notin D_{r_1}$ but
            we know that $D_{r_1} = D_{r_2}$ therefore we have a contradiction.
            With the same type of arguments we can show that $r_2>r_1$ cannot be either.
            So must be $r_1 = r_2$.
            \item [(2)] By definition $i(0) = D_0$ and $i(1) = D_1$.
            \item [(3)]
            \begin{itemize}
                \item [(a)] Let $t \in D_{r+s}$ so $t > r+s$. Since $D_{r+s}$ is a
                Dedekin cut then we know there is $a \in D_{r+s}$ such that $t>a>r+s$
                then $a-s>r$ so $a-s \in D_{r}$. Also, we can write that
                $t = (a-s) + (t -a + s)$, and since $t>a$ then $t-a>0$ so $t-a+s>s$
                which means that $t-a+s \in D_s$. Therefore we see that $t \in D_r + D_s$.  

                Let $t \in D_r + D_s$ then $t = x + y$ where $x \in D_r$ and $y \in D_s$.
                Then $x>r$ and $y>s$ and by adding both inequalities we have that
                $x+y>r+s$ which means that $x+y \in D_{r+s}$. It follows that
                $D_r + D_s \subseteq D_{r+s}$.
                
                Therefore must be that  $D_{r+s} = D_r + D_s$ which means that\\
                $i(r+s) = i(r)+i(s)$.
                \item [(b)] By definition $i(-r) = D_{-r}$ and from what we proved in
                Exercise 1.7.1 we have that $D_{-r} = -D_{r} = -i(r)$.
                
                Therefore $i(-r) = -i(r)$.
\cleardoublepage
                \item [(c)]Let $r>0$ and $s>0$.
                
                If $t \in D_{rs}$ then $t > rs$. Since $D_{rs}$ is a Dedekin cut then we
                know there is $a \in D_{rs}$ such that $t>a>rs$. Also, we can write
                $t = \frac{a}{s}\cdot\frac{ts}{a}$ and from the inequality we have that 
                $\frac{a}{s}>r$ so $\frac{a}{s} \in D_{r}$.
                On the other hand $t>a$ so $\frac{ts}{a}>s$ which means that
                $\frac{ts}{a} \in D_s$. Therefore $t \in D_rD_s$ so
                $D_{rs} \subseteq D_rD_s$.

                Let now $t \in D_rD_s$ so $t = xy$ for some $x \in D_r$ and $y \in D_s$.
                Also, we know that $x >r>0$ and $y>s>0$ then by multiplying the
                inequalities we have that $t = xy>rs$ it follows that $t \in D_{rs}$ and 
                that $D_rD_s \subseteq D_{rs}$. Therefore $D_{rs} = D_rD_s$.

                Now let $r>0$ and $s<0$ which means that $D_r \geq D_0$ and $D_s < D_0$.
                Then 
                $$D_{rs} = D_{r(-(-s))} = -D_{r(-s)} = -[D_rD_{-s}] = -[D_r(-D_s)] = D_rD_s$$
                where we used first the Exercise 1.7.1 part (1) then the result we got
                for the case where $r>0$ and $s>0$ and last the definition of
                multiplication of Dedekin cuts.
                
                In the case of $r<0$ and $s>0$ we have in the same way that 
                $$D_{rs} = D_{(-(-r))s} = -D_{(-r)s} = -[D_{-r}D_s] = -[(-D_r)D_s] = D_rD_s$$
                
                And finally in the case of $r<0$ and $s<0$ we have that 
                \begin{align*}
                    D_{rs} &= D_{(-(-r))(-(-s))} = D_{(-r)(-s)} = D_{-r}D_{-s} = [(-D_r)(-D_s)] \\
                           &= D_rD_s
                \end{align*}

                Therefore $D_{rs} = i(rs) = i(r)i(s) = D_rD_s$.

                \item [(d)] By definition $i(r^{-1}) = D_{r^{-1}}$ and by what we proved
                in Exercise 1.7.1 part (2) we have that $D_{r^{-1}} = [D_{r}]^{-1}$.

                Therefore $i(r^{-1}) = D_{r^{-1}} = [D_{r}]^{-1} = [i(r)]^{-1}$.

                \item [(e)] ($\rightarrow$) Let $x \in D_s$ then by definition $s<x$
                but we know that $r<s<x$ so $x \in D_r$ which
                means that $D_s \subsetneqq D_r$ then $D_r < D_s$ and $i(r)<i(s)$.

                ($\leftarrow$) If $i(r)<i(s)$ then $D_r < D_s$ and $D_s \subsetneqq D_r$.
                We will show by contradiction that $r<s$.
                Suppose that $s<r$ then $r \in D_s$ but then $D_r \subsetneqq D_s$ so 
                $D_s < D_r$, but we know that $D_r < D_s$ so we have a contradiction.
                
                Now suppose that $r=s$ then $D_r = D_s$ but know that $D_r < D_s$ so
                we have another contradiction.

                Therefore by Trichotomy must be the case that $r<s$.
            \end{itemize}
        \end{itemize}
    \end{proof}
    
\end{document}






















