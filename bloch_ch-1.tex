\documentclass[11pt]{article}
\usepackage{amssymb}
\usepackage{amsthm}
\usepackage{enumitem}
\usepackage{amsmath}
\usepackage{bm}
\usepackage{adjustbox}
\usepackage{mathrsfs}
\usepackage{graphicx}
\usepackage{siunitx}
\usepackage[mathscr]{euscript}

\title{\textbf{Solved selected problems of Real Numbers and Real Analysis - Bloch}}
\author{Franco Zacco}
\date{}

\addtolength{\topmargin}{-3cm}
\addtolength{\textheight}{3cm}

\newcommand{\N}{\mathbb{N}}
\newcommand{\Z}{\mathbb{Z}}
\theoremstyle{definition}
\newtheorem*{solution*}{Solution}

\begin{document}
\maketitle
\thispagestyle{empty}

\section*{Chapter 1 - Construction of the Real Numbers}

	\begin{proof}{\textbf{1.2.1}}
        To prove the uniqueness of $\cdot:\N\times\N \rightarrow \N$, we
        suppose that there are two operations $\cdot$ and $\odot$ on $\N$ that
        satisfy the two properties of the theorem. Let
        $$G = \{x\in\N~|~n\cdot x = n \odot x\text{ for all }n \in\N \}$$
        we want to prove that $G = \N$, which will imply that $\cdot$ and $\odot$
        are the same operation. It is clear that $G \subseteq \N$. By part (a)
        applied to each of $\cdot$ and $\odot$ we see that
        $n\cdot 1 = n = n \odot 1$ for all $n \in \N$ and then $1 \in G$.
        
        Now let $q \in G$ so $n \cdot q = n \odot q$ and then from part (b) we
        have that $n \cdot s(q) = (n\cdot q) + n = (n\odot q) + n = n\odot s(q)$
        hence $s(q) \in G$.

        Finally, we use part (c) of Peano Postulates to conclude that $G = \N$
    \end{proof}
	\begin{proof}{\textbf{1.2.2}}
        \begin{itemize}
        \item [(2)] Let 
        $$G = \{c \in \N ~|~ (a+b)+c=a+(b+c) \text{ for all }a,b\in\N\}$$
        We will show that $G = \N$ which will imply the desired result. Clearly
        $G\subseteq\N$. To show that $1 \in G$, let $j, k \in \N$ so
        $(j+k)+1=s(j+k)=j+s(k)=j+(k+1)$ then $1 \in G$, where we used from
        Theorem 1.2.5 that $s(n+m) = n + s(m)$.

        Now let $r \in G$. Let $j,k \in G$ and suppose $(j+k)+s(r) = j + (k+s(r))$
        by Theorem 1.2.5 we know that $(j+k)+s(r)=s((j+k)+r)$ and since
        $r \in G$ then $s((j+k)+r)=s(j+(k+r))$ then $s(j+(k+r))=j+s(k+r)=j+(k+s(r))$
        therefore $s(r) \in G$ and $G =\N$.
        \item [(3)] Let
        $$H = \{a \in\N ~|~ a+1=1+a=s(a)\}$$
        We will show that $H = \N$ which will imply the desired result. Clearly
        $H\subseteq\N$. If $a=1$ replacing we have that $1+1=1+1=s(1)$ then
        $1\in H$.
        
        Now let $r \in H$ and suppose $s(r)+1=1+s(r)$ then
        \begin{align*}
            s(r)+1 &= (r+1)+1 &\text{because part (a) of Theorem 1.2.5}\\
                &= (1+r)+1 &\text{because } r \in H \\
                &= 1+(r+1) &\text{because } a+(b+c) = (a+b)+c \\
                &= 1+s(r) &\text{because part (a) of Theorem 1.2.5}
        \end{align*}
        Therefore $H = \N$.
        \item [(4)] Let
        $$G = \{a\in\N ~|~ a+b=b+a \text{ for all }b\in\N\}$$
        We will show that $G = \N$ which will imply the desired result. It is
        clear that $G\subseteq\N$. To show that $1 \in G$, let $j\in \N$ so
        $1+j=j+1$ because we proved that in part (3) then $1 \in G$.

        Now let $r\in G$ and suppose that $j +s(r) = s(r)+j$ then
        \begin{align*}
            j + s(r) &= s(j+r) &\text{because part (b) of Theorem 1.2.5} \\
                &= s(r+j) &\text{because }r\in G \\
                &= r + s(j) &\text{because part (b) of Theorem 1.2.5} \\
                &= r + (j + 1) &\text{because part (a) of Theorem 1.2.5} \\
                &= r + (1 + j) &\text{because part (3)} \\
                &= (r + 1) + j &\text{because associative law for addition} \\
                &= s(r) + j &\text{because part (a) of Theorem 1.2.5}
        \end{align*}
        Therefore $G = \N$.
        \item [(7)] Let
        $$G = \{a \in N ~|~ 1\cdot a = a \cdot 1 = a\}$$
        We will show that $G = \N$ which will imply the desired result. It is
        clear that $G\subseteq\N$. If $a=1$ replacing we have that
        $1\cdot 1 = 1\cdot 1 = 1$ because of part (a) in Theorem 1.2.6 then
        $1 \in G$.

        Now let $r \in G$ and suppose $s(r)\cdot 1 = 1 \cdot s(r) = s(r)$ then
        \begin{align*}
            s(r) \cdot 1 &= s(r) &\text{because part(a) of Theorem 1.2.6}\\
                &= r + 1 &\text{because part(a) of Theorem 1.2.5}\\
                &= (r \cdot 1) + 1 &\text{because }r \in G\\
                &= (1 \cdot r) + 1  &\text{because }r \in G\\
                &= 1 \cdot s(r) &\text{because part (b) of Theorem 1.2.6}
        \end{align*}
        Therefore $G = \N$.
\cleardoublepage
        \item [(8)] Let
        $$H = \{c \in \N ~|~ (a+b)c=ac+bc \text{ for all }a,b \in \N\}$$
        We want to show that $H = \N$ which will imply the desired result. It 
        is clear that $H\subseteq\N$. To show that $1 \in H$ let $j,k \in \N$
        then $(j+k)\cdot 1 = j+k = j\cdot 1 +k \cdot 1$ because what we proved
        in part (7), then $1 \in H$.

        Now let $r \in H$, let $j,k \in \N$ and suppose
        $(j+k)\cdot s(r) = j\cdot s(r) + k\cdot s(r)$ then
        \begin{align*}
            (j+k)\cdot s(r) &= ((j+k)\cdot r) + (j+k) &\text{because part (b) of Theorem 1.2.6}\\
                &= (j\cdot r + k \cdot r) + (j+k) &\text{because }r \in H\\
                &= (j\cdot r + j) + (k\cdot r + k) &\text{because Commutative and Associative law} \\
                &= j\cdot s(r) + k \cdot s(r) &\text{because part (b) of Theorem 1.2.6}
        \end{align*}
        Therefore $s(r) \in H$ and $H = \N$.
        \end{itemize}
        \item [(9)] Let
        $$H = \{a \in \N ~|~ ab=ba \text{ for all }b \in \N\}$$
        We want to show that $H = \N$ which will imply the desired result. It 
        is clear that $H\subseteq\N$. Also $1 \in H$ because what we proved in
        part (7).

        Now let $r \in H$, let $k \in \N$ and suppose $s(r)\cdot k = k \cdot s(r)$
        then
        \begin{align*}
            s(r) \cdot k &= (r+1) \cdot k & \text{because part (a) of Theorem 1.2.5} \\
                &= r\cdot k + 1\cdot k & \text{because Distributive law} \\
                &= k\cdot r + k & \text{because }r \in H \\
                &= k \cdot s(r) & \text{because part (b) of Theorem 1.2.6}
        \end{align*}
        Therefore $s(r) \in H$ and $H = \N$.
        \item [(10)] Let
        $$H = \{c \in \N ~|~ c(a+b)=ca+cb \text{ for all }a,b \in \N\}$$
        We want to show that $H = \N$ which will imply the desired result. It 
        is clear that $H\subseteq\N$. Let $j,k \in \N$, if $c=1$ then
        $1\cdot(j+k) = j+k = 1\cdot j + 1\cdot k$ so $1 \in H$.

        Now let $r \in H$ and lets suppose that
        $s(r)\cdot(j+k)=s(r)\cdot j + s(r) \cdot k$ then
        \begin{align*}
            s(r)\cdot(j+k) &= (j+k)\cdot s(r) & \text{because part (9)}\\
                &= j\cdot s(r) + k\cdot s(r) & \text{because right-hand side Distributive law}\\
                &= s(r) \cdot j + s(r) \cdot k & \text{because part (9)}
        \end{align*}
        Therefore $s(r) \in H$ and $H = \N$.
\cleardoublepage
        \item [(11)] Let
        $$H = \{c \in \N ~|~ (ab)c=a(bc) \text{ for all }a,b \in \N\}$$
        We want to show that $H = \N$ which will imply the desired result. It 
        is clear that $H\subseteq\N$. Let $j,k \in \N$, if $c = 1$ then
        $(j\cdot k)\cdot 1 = j\cdot k = j\cdot (k) = j\cdot (k\cdot 1)$ then $1 \in H$.
        
        Now let $r \in H$ and suppose $(j\cdot k)\cdot s(r) = j\cdot(k\cdot s(r))$
        then
        \begin{align*}
            (j \cdot k) \cdot s(r) &= ((j \cdot k) \cdot r ) + (j \cdot k) & \text{because part (b) of Theorem 1.2.6}\\
                &= (j \cdot (k \cdot r)) + (j \cdot k) & \text{because }r \in H \\
                &= j \cdot ((k\cdot r) + k) & \text{because Distributive law} \\
                &= j \cdot (k \cdot s(r)) & \text{because part (b) of Theorem 1.2.6}
        \end{align*}
        Therefore $s(r) \in H$ and $H = \N$.
        \item [(13)] Let $ab=1$ there are a set of cases that we should check
        \begin{itemize}
            \item if $a = 1$ and $b \neq 1$ then $1 \cdot b = b$ but we said
            that $ab = 1$ then $b$ must be equal to 1 which is a contradiction.
            \item if $a \neq 1$ and $b = 1$ then $a \cdot 1 = a$ but we said
            that $ab = 1$ then $a$ must be equal to 1 which is a contradiction.
            \item if $a \neq 1$ and $b \neq 1$ then because of Lemma 1.2.3
            there is a unique $t \in \N$ such that $b= s(t)$ so 
            $a \cdot b = a \cdot s(t) = a \cdot t + a = 1$ and because of part
            (5) this can't be true.
        \end{itemize}        
        Now let $a=b=1$ then $a\cdot b=1 \cdot 1 = 1$ which is what we wanted.
    \end{proof}
	\begin{proof}{\textbf{1.2.3}} Let $p_1, p_2 \in\N$ where $p_1 \neq p_2$
        such that $a +p_1 = b$ and $a + p_2 = b$ then $a + p_1 = a + p_2$ and
        because of the Cancellation law we have that $p_1 = p_2$ which is a
        contradiction. Therefore there is a unique $p \in \N$ such that
        $a + p = b$. 
    \end{proof}
	\begin{proof}{\textbf{1.2.4}}
        \begin{itemize}
        \item [(1)] Since $a = a$ then by definition of the operation $\leq$
        is clear that $a \leq a$.

        Now let $a < a$ then there is a $p\in\N$ such that $a = a + p$ but
        this is not possible because of part (6) of Theorem 1.2.7 then it's
        a contradiction and therefore $a \nless a$.

        Let $b = a + 1$ then if $a < b$ there is a $p \in \N$ such that
        $b = a + p$ but then $a + 1 = a + p$ and then $p = 1$, so we found
        $p = 1$ for which $a < b = a + 1$.
\cleardoublepage
        \item [(3)] If $a < b$ and $b < c$ then there is a $p \in \N$ and a
        $q \in \N$ such that $b = a + p$ and $c = b + q$ so replacing variable
        $b$ we have that $c = (a + p) + q = a + (p + q)$ now naming $k = p + q$
        we have $c = a + k$ and then by definition $a < c$.
        
        If $a \leq b$ and $b <c$ then either $a < b$ or $a = b$, the first case
        was already proven so we focus on the second one. We also have that
        $b < c$ then by definition there is a $p \in \N$ such that $c = b + p$
        but $a = b$ then replacing $c = a + p$ and by definition $a<c$.

        If now, $a < b$ and $b \leq c$ we have that either $b < c$ or $b = c$
        the first case was already proven so we focus on the second one.
        Given that $a < b$ there is a $p \in \N$ such that $b = a +p$ but if
        $b = c$ then $c = a + p$ which by definition says that $a < c$.

        Finally, if $a \leq b$ and $b \leq c$ then either $a<b$ or $a=b$ and
        $b<c$ or $b =c$, the last combination we have to prove is the case
        where $a=b$ and $b=c$ then $a=b=c$ so it's clear that $a \leq c$.
        \item [(4)] Let $a < b$ then by definition there is a $p \in \N$ such
        that $b = a + p$ then $b + c = (a + p) + c$ because of part (1) of
        Theorem 1.2.7, and by the Commutative and Associative law we have that
        $b + c = (a + c) + p$ which by definition says that $a + c < b + c$.

        If $a + c < b +c$ then by definition there is a $p \in \N$ such that
        $b + c = (a + c) + p$ and by the Commutative and Associative law we
        have that $b + c = (a + p) + c$ and because of part (1) of Theorem
        1.2.7 we have that $b = a + p$ which by definition says that $a < b$.
        \item [(5)] Let $a<b$ then by definition there is a $p \in \N$ such
        that $b = a + p$ and because of part (12) of Theorem 1.2.7 we have that
        $bc = (a+p)c = ac + pc$ where we also applied the Distributive law
        now naming $k = pc$ we have that $bc = ac + k$ which by definition
        means that $ac < bc$.

        Let $ac < bc$ and suppose $a\nless b$ then by the Trichotomy law either
        $a>b$ or $a=b$. If $a>b$ then because what we proved $ac>bc$ which is
        a contradiction to $ac < bc$. If $a = b$ then $ac = bc$ because of part
        (12) of Theorem 1.2.7. and it's another contradiction to the fact that
        $ac < bc$. Then must happen that $a<b$.
        \item [(11)] Let $a < b$ and suppose $b < a+1$ then $a < b < a + 1$ but
        this cannot happen because of part (9) so it must happen because of the
        Trichotomy law that $a + 1 \leq b$.

        Now let $a + 1 \leq b$ and suppose that $a = b$ then $a + 1 \leq b = a$
        which cannot be because of part (1) of this Theorem, so lets suppose
        that $a > b$ then $a + 1 < a$ because of part (3) of this Theorem but
        that cannot be true because of part (1) of this Theorem. Therefore
        it must be that $a < b$.
        \end{itemize}
    \end{proof}
    \begin{proof}{\textbf{1.2.5}}
        Let $a+a = b+b$ then because of part (7) of Theorem 1.2.7 we can write
        that $1\cdot a + 1 \cdot a = 1 \cdot b + 1 \cdot b$ and because of the
        Distributive law we have that $a\cdot (1 + 1) = b\cdot (1 + 1)$ if we
        name $c = 1 + 1$ then $ac=bc$ and because of part (12) of Theorem 1.2.7
        we have that $a = b$.
    \end{proof}
    \begin{proof}{\textbf{1.2.6}}
        Let
        $$H = \{n \in \N ~|~1 \leq n \leq b\} \cup\{n \in \N ~|~ b+1 \leq n\}$$
        We want to show that $H = \N$. It is clear that $H\subseteq\N$. Because
        $1 \in \{n \in \N ~|~1 \leq n \leq b\}$ by definition then $1 \in H$.\\
        Now let $r \in H$, we want to show that $r+1 \in H$, if $r=b$ then
        $$r+1=b+1 \in \{n \in \N ~|~ b+1 \leq n\}$$
        if $r < b$ then because of part (11) of Theorem 1.2.9 $r+1 \leq b$ then
        $$r+1 \in \{n \in \N ~|~1 \leq n \leq b\}$$
        if $b < r$ then because of part (11) or Theorem 1.2.9 $b+1 \leq r$
        also because of part (1) of Theorem 1.2.9 $r < r+1$ then because
        of part (3) of Theorem 1.2.9 we have that $b+1 < r+1$ then
        $$r+1 \in \{n \in \N ~|~ b+1 \leq n\}$$
        Therefore $r+1 \in H$ and $H = \N$.\\
        Now Let
        $$G = \{n \in \N ~|~1 \leq n \leq b\} \cap\{n \in \N ~|~ b+1 \leq n\}$$
        Suppose there is an $r\in G$. We will derive a contradiction.
        Then it must happen that $1\leq r \leq b$ and that $b +1 \leq r$ but
        then because of part (3) of Theorem 1.2.9 it must happen that 
        $b+1 \leq b$ which is a contradiction to the part (6) of Theorem 1.2.9.
        Therefore there is no $r \in G$.
    \end{proof}
    \begin{proof}{\textbf{1.2.7}}
        \begin{itemize}
        \item [(1)] Let
        $$H = \{n \in \N ~|~ a+n \in A \text{ for all }a\in A\}$$
        We want to show that $H = \N$ which will imply the desired result. It 
        is clear that $H\subseteq\N$. To show that $1 \in H$ let $b \in A$ then
        $b + 1 \in A$ by definition of $A$ and then $1 \in H$.

        Now let $r \in H$ then $b + r \in A$ for some $b \in A$ and by
        definition of $A$ we have that $(b + r) + 1 = b + (r + 1)\in A$ then
        $r +1 \in H$ and therefore $H = \N$.
        \item [(2)] Let $H = \{x \in \N ~|~ x \geq a\}$ and let $r \in H$ then
        $r \geq a$ so it must happen that $r = a$ or $r > a $ in the first
        case it is clear that $r = a \in A$ in the second case by definition
        there is a $p \in \N$ such that $r = a + p$ and we know because of the
        part (1) that $a + p \in A$. Therefore it must happen that
        $H \subseteq A$.
        \end{itemize}
    \end{proof}
    \begin{proof}{\textbf{1.2.8}}
        We want to prove that there is an inverse function for $f$. We have a
        set $\N'$ with an element $1'\in\N'$ and a function
        $s':\N'\rightarrow \N'$ that satisfy the Peano Postulates so we could
        say because Theorem 1.2.4 that there is a function
        $g:\N' \rightarrow \N$ such that $g(1') = 1$ and $g \circ s' = s \circ g$.
        Now we have to check that $g$ is the inverse of $f$.\\
        Let
        $$G = \{n \in \N ~|~ g(f(n)) = n\}$$
        We want to show that $G = \N$. It  is clear that $G\subseteq\N$.
        To show that $1 \in G$ we do $g(f(1)) = g(1') = 1$ which means that
        $1 \in G$.\\
        Now Let $r \in G$ we want to show that $r+1 \in G$ then
        \begin{align*}
            g(f(r+1)) &= g(f(s(r))) & \text{by definition of }r+1\\
                &= g(s'(f(r))) & \text{because }f \circ s = s' \circ f\\
                &= s(g(f(r))) & \text{because }g \circ s' = s \circ g\\
                &= s(r) = r + 1 & \text{because }r \in G
        \end{align*}
        Therefore $r+1 \in G$ and $G = \N$.
        In the same way, let
        $$H = \{n \in \N' ~|~ f(g(n)) = n\}$$ 
        We want to show that $H = \N'$. It  is clear that $H\subseteq\N'$.
        To show that $1' \in H$ we do $f(g(1')) = f(1) = 1'$ which means that
        $1' \in H$.\\
        Now let $r' \in H$ we want to show that $r' +1 \in H$ then
        \begin{align*}
            f(g(r'+1)) &= f(g(s'(r'))) & \text{by definition of }r'+1\\
                &= f(s(g(r'))) & \text{because }g \circ s' = s \circ g\\
                &= s'(f(g(r'))) & \text{because }f \circ s = s' \circ f\\
                &= s'(r') = r' + 1 & \text{because }r' \in H
        \end{align*}
        Therefore $r'+1 \in H$ and $H = \N'$.\\
        Finally, $g$ is the inverse of $f$ hence $f$ is bijective, which is
        what we wanted.
    \end{proof}
    \begin{proof}{\textbf{1.3.1}}
        \begin{itemize}
            \item [(1)] We want to prove that $\approx$ is an equivalence relation so we
            must prove that it is reflexive, symmetric and transitive. Let
            $(a,b),(c,d) \in \N \times \N$. Given that $a^2b=a^2b$ then
            $(a,b) \approx (a,b)$. Therefore $\approx$ is reflexive. \\
            Suppose that $(a,b) \approx (c,d)$ then $a^2d=c^2b$ but also $c^2b=a^2d$
            hence $(c,d) \approx (a,b)$ and therefore $\approx$ is symmetric. \\
            Now suppose that also $(e,f) \in \N \times \N$ and $(c,d) \approx (e,f)$ 
            then $c^2f = e^2d$ but we know that $a^2d=c^2b$ multiplying this last
            equation on both sides with $f$ and doing a few re-arrangements we have that
            $a^2df = c^2fb = e^2db$ then $a^2f=e^2b$ which means that
            $(a,b) \approx (e,f)$. Therefore $\approx$ is transitive.\\
            Finally since $\approx$ is reflexive, symmetric and transitive then
            $\approx$ is an equivalence relation on $\N \times \N$.
            \item [(2)] The elements of the equivalence class $[(2,3)]$ are
            $$[(2,3)] = \{(x,y)\in\N\times\N~|~ 4y = (x^2)3\}$$
        \end{itemize}
    \end{proof}
    \begin{proof}{\textbf{1.3.2}}
            We want to complete the proof by showing that $\sim$ is transitive. Let
            $(e,f) \in \N \times \N$ and $(c,d) \sim (e,f)$ then $c+f = d+e$ also we
            know that $a+d = b+c$ and adding to both sides $f$ we get that
            $a+d+f = b+c+f = b+d+e$ then $a+f = b+e$ because of Theorem 1.2.7 part (1).
            Therefore $(a,b)\sim(e,f)$.\\
            Finally since $\sim$ is reflexive, symmetric and transitive then
            $\sim$ is an equivalence relation on $\N \times \N$.
    \end{proof}
    \begin{proof}{\textbf{1.3.3}}
        Let's prove first that $-$ is well-defined for $\Z$. Let
        $(a,b),(x,y) \in \N \times \N$ and suppose that $[(a,b)] = [(x,y)]$ then
        $(a,b)\sim(x,y)$ so $a+y = b+x$ but if we write $b+x=a+y$ we have that
        $(b,a) \sim (y,x)$ therefore $-[(a,b)]=[(b,a)]=[(y,x)]=-[(x,y)]$.\\
        Now let's prove that $\cdot$ is well-defined for $\Z$. Let
        $(a,b),(c,d),(x,y),(z,w) \in \N \times \N$ and suppose $[(a,b)] = [(x,y)]$ and
        $[(c,d)] = [(z,w)]$ so by hypothesis $(a,b) \sim (x,y)$ and $(c,d) \sim (z,w)$
        then $a+y=b+x$ and $c+w=d+z$.
        Taking into account this let's do
        \begin{align*}
            &(ac + bd + xw + yz) + (xc + yc + xd + yd) = \\
            &= c(a + y + x) + d(b + x + y) + xw + yz \\
            &= c(b + x + x) + d(a + y + y) + xw + yz \\
            &= bc + xc + xc + ad + yd + yd + xw + yz \\
            &= ad + bc + x(c + c + w) + y(d + d + z) \\
            &= ad + bc + x(c + d + z) + y(d + c + w) \\
            &= (ad + bc + xz + yw) + (xc + yc + xd + yd)
        \end{align*}
        which proves that $ac + bd + xw + yz  = ad + bc + xz + yw$ and therefore
        $[(ac+bd,ad+bc)] = [(xz+yw,xw+yz)]$.
    \end{proof}
    \begin{proof}{\textbf{1.3.4}}
    \begin{itemize}
        \item [(1)]
        ($\rightarrow$) If $[(a,b)]=\hat{0}$ then $[(a,b)] = [(1,1)]$ because of the
        definition of $\hat{0}$ then $(a,b) \sim (1,1)$ so $a+1=b+1$ because of the
        Cancellation Law of $\N$ we have that $a=b$.\\
        ($\leftarrow$) If $a=b$ then adding to both sides $1$ we have that $a+1=b+1$
        then by the Definition 1.3.1 $(a,b)\sim(1,1)$ therefore $[(a,b)]=[(1,1)]$.
        \item [(2)]
        ($\rightarrow$) If $[(a,b)]=\hat{1}$ then $[(a,b)] = [(1+1,1)]$ because of the
        definition of $\hat{1}$ then $(a,b) \sim (1+1,1)$ so $a+1=(b+1)+1$ because of
        the Cancellation Law of $\N$ we have that $a=b+1$.\\
        ($\leftarrow$) If $a=b+1$ then adding to both sides $1$ we have that
        $a+1=(b+1)+1=b+(1+1)$ then by the Definition 1.3.1 $(a,b)\sim(1+1,1)$
        therefore $[(a,b)]=[(1+1,1)]$.
        \item [(3)]
        First let's prove that $[(a,b)]=[(n,1)]$ for some $n \in \N$ such that
        $n \neq 1$ is and only if $a>b$.\\
        ($\rightarrow$) Let $[(a,b)]=[(n,1)]$ for some $n \in \N$ where $n \neq 1$ then
        $a+1=b+n$ and given that $n\in\N$ and $n \neq 1$ then $n$ can be written as
        $n = q +1$ for some $q\in\N$ then $a+1=b+q+1$ and by the Cancellation law of
        $\N$ we have that $a=b+q$ which by definition means that $a>b$.\\
        ($\leftarrow$) Let $a>b$ by definition this means that $a = b +m$ where $m\in\N$
        then if we add $1$ to both sides of the equation we have that
        $a + 1 = b + m + 1$ then by naming $n = m + 1$ where $n \in \N$ we have that
        $a+1=b+n$ and therefore $[(a,b)]=[(n,1)]$ where $n \neq 1$.\\
        Now Let's prove that $a>b$ if and only if $[(a,b)]>\hat{0}$\\
        ($\rightarrow$) Let $a>b$  if we add on both sides of the equation $1$ we get
        that $a+1>b+1$ then this means that $[(a,b)]>\hat{0}=[(1,1)]$.\\
        ($\leftarrow$) Let $[(a,b)]>\hat{0}=[(1,1)]$ then $a+1>b+1$ and because of
        Theorem 1.2.9 part (4) we have that $a>b$.
        \item [(4)]
        First let's prove that $[(a,b)]=[(1,m)]$ for some $m \in \N$ such that
        $m \neq 1$ if and only if $a<b$.\\
        ($\rightarrow$) Let $[(a,b)]=[(1,m)]$ for some $m \in \N$ where $m \neq 1$ then
        $a+m=b+1$ and given that $m\in\N$ and $m \neq 1$ then $m$ can be written as
        $m = q +1$ for some $q\in\N$ then $a+q+1=b+1$ and by the Cancellation law of
        $\N$ we have that $a+q=b$ which by definition means that $a<b$.\\
        ($\leftarrow$) Let $a<b$ by definition this means that $b = a +n$ where $n\in\N$
        then if we add $1$ to both sides of the equation we have that
        $b + 1 = a + n + 1$ then by naming $m = n + 1$ where $m \in \N$ we have that
        $a+m=b+1$ and therefore $[(a,b)]=[(1,m)]$ where $m \neq 1$.\\
        Now Let's prove that $a<b$ if and only if $[(a,b)]<\hat{0}$\\
        ($\rightarrow$) Let $a<b$  if we add on both sides of the equation $1$ we get
        that $a+1<b+1$ then this means that $[(a,b)]<\hat{0}=[(1,1)]$.\\
        ($\leftarrow$) Let $[(a,b)]<\hat{0}=[(1,1)]$ then $a+1<b+1$ and because of
        Theorem 1.2.9 part (4) we have that $a<b$.
    \end{itemize}
    \end{proof}
    \begin{proof}{\textbf{1.3.5}}
    \begin{itemize}
    \item [(1)]
    Using the definition of addition of integers we see that
    \begin{align*}
        (x+y)+z &= ([(a,b)]+[(c,d)])+[(e,f)] \\
                &= [(a+c,b+d)] + [(e,f)] \\
                &= [((a+c)+e,(b+d)+f)] \\
                &= [(a+(c+e),b+(d+f))] \\
                &= [(a,b)] + [(c+d,d+f)] \\
                &= [(a,b)] + ([(c,d)] + [(e,f)]) \\
                &= x+(y+z)
    \end{align*}
    where the middle equality holds because of the Associative law of $\N$.
    \item [(3)]
    We want to prove that $x+\hat{0} = x$ by arriving to a contradiction. Let us suppose
    $x + \hat{0} \neq x$ then this means that $[(a,b)]+[(1,1)] = [(c,d)]$ where
    $[(c,d)] \neq [(a,b)]$ then $[(a+1,b+1)] = [(c,d)]$ so $a+1+d=b+1+c$ and by the
    Cancellation of $\N$ we have that $a+d=b+c$ then $[(a,b)]=[(c,d)]$ which is a
    contradiction and therefore $x+\hat{0} = x$.
    \item [(4)]
    Let $x=[(a,b)]$ then from the equation $(a+b)+1=(b+a)+1$ we have that
    $[(a+b,b+a)]=[(1,1)]$ and thus $[(a,b)]+[(b,a)] = [(1,1)]$ therefore $x+(-x)=\hat{0}$. 
    \item [(5)]
    Let $x = [(a,b)]$, $y = [(c,d)]$ and $z = [(e,f)]$ then
    \begin{align*}
        (xy)z &= ([(a,b)]\cdot[(c,d)])\cdot[(e,f)] \\
              &= [(ac + bd,ad + bc)] \cdot[(e,f)] \\
              &= [((ac + bd)e+(ad + bc)f, (ac + bd)f+(ad + bc)e)] \\
              &= [(ace + bde+adf + bcf, acf + bdf+ ade + bce)] \\
              &= [(a(ce + df) + b(de + cf), a(cf + de) + b(df + ce))] \\
              &= [(a,b)] \cdot [(ce + df, de + cf)] \\
              &= [(a,b)] \cdot ([(c,d)] \cdot [(e,f)]) = x(yz)
    \end{align*}
    \item [(6)]
    Let $x = [(a,b)]$ and $y = [(c,d)]$ then
    \begin{align*}
        xy &= [(a,b)]\cdot[(c,d)] \\
           &= [(ac+bd,ad+bc)] \\
           &= [(ca+db,cb+da)] \\
           &= [(c,d)]\cdot[(a,b)] = yx    
    \end{align*}
    Where the middle equality is true because of the Commutative Law for Addition and
    Multiplication.
    \item[(7)]
    Let $a,b\in\N$ then the equation $a(1+1)+b(1)+b=b(1+1)+a(1)+a$ is true and if we
    let $x=[(a,b)]$ then the equation means that
    $$[(a(1+1) + b(1),b(1+1) + a(1))] = [(a,b)]$$
    then because of the definition of the $\cdot$ operation we have that
    $$[(a,b)] \cdot [(1+1,1)] = [(a,b)]$$
    therefore $x \cdot \hat{1} = x$.
    \item [(8)]
    Let $x = [(a,b)]$, $y = [(c,d)]$ and $z = [(e,f)]$ then
    \begin{align*}
        x(y+z) &= [(a,b)]\cdot([(c,d)] + [(e,f)]) \\
               &= [(a,b)]\cdot[(c+e,d+f)] \\
               &= [(a(c+e)+b(d+f),a(d+f)+b(c+e))] \\
               &= [(ac+ae+bd+bf,ad+af+bc+be)] \\
               &= [((ac+bd)+(ae+bf),(ad+bc)+(af+be))] \\
               &= [(ac+bd,ad+bc)] + [(ae+bf,af+be)] \\
               &= [(a,b)]\cdot[(c,d)] + [(a,b)]\cdot[(e,f)] = xy + xz
    \end{align*}
    \item [(10)]
    Let us first prove that there is no way that two of $x<y$, $x=y$ or $x>y$ can be
    true at the same time. \\
    Let $x = [(a,b)]$ and $y = [(c,d)]$ and let's suppose that $x<y$ and $x=y$ are both
    true then $x<x$ which means that $[(a,b)]<[(a,b)]$ and thus $a+b<b+a$
    by the Cancellation law $a<a$ but that is a contradiction to Theorem 1.2.9 part (1).\\
    In the same way let's suppose that $x>y$ and $x=y$ are both true then $x>x$, which
    leads to the same result as before which is a contradition.\\
    Finally, let's suppose that $x<y$ and $y<x$ are both true then from the first
    inequality we have that $[(a,b)]<[(c,d)]$ then $a+d<b+c$, and from the last inequality
    we have that $[(c,d)]<[(a,b)]$ then $c+b<a+d$ and by applying the Transitive law of
    $\N$ we have that $a+d<a+d$ thus $a<a$ and we have already proven that this is a
    contradiction.\\
    Therefore no two of $x<y$, $x=y$ and $x>y$ can be true at the same time.\\
    Now let's prove that one of them is always true. \\
    Suppose $x,y \in \Z$ then $x=[(a,b)]$ and $y=[(c,d)]$ also it must be true that
    $a+d<b+c$ or $a+d=b+c$ or $a+d>b+c$ because of Trichotomy of $\N$ if $a+d<b+c$ is
    true then that means that $[(a,b)]<[(c,d)]$, if $a+d=b+c$ then $[(a,b)]=[(c,d)]$ or
    if $a+d>b+c$ then $[(a,b)]>[(c,d)]$.
    \item [(11)]
    Let $x=[(a,b)]$, $y=[(c,d)]$ and $z=[(e,f)]$ then if $x<y$ and $y<z$ that means 
    that $a+d<b+c$ and from the second inequality we have that $c+f<d+e$ then
    by definition $b+c = (a+d) + p$ and $d+e = (c+f)+q$ where $p,q \in \N$ then summing
    both equations we have that
    \begin{align*}
        (b+c)+(d+e) &= (a+d)+p+(c+f)+q \\
        b+e &= (a+f)+(p+q)
    \end{align*}
    If we name $k=p+q$ then by definition $a+f<b+e$ and therefore $[(a,b)]<[(e,f)]$ thus
    $x<z$.
    \item [(13)]
    Let $x=[(a,b)]$, $y=[(c,d)]$ and $z=[(e,f)]$ then if $x<y$ this means that $a+d<b+c$
    and by definition $b+c=(a+d)+p$ where $p\in\N$ also we know that $\hat{0}<z$ so
    $1+f<1+e$ and also by definition this means that $e = f + q$ where $q \in \N$.
    Multiplying both sides of $b+c=(a+d)+p$ with $e$ we get that
    $$e(b+c)=e(a+d)+ep$$
    And doing the same with $f$ we get that
    $$f(a+d)+fp=f(b+c)$$
    Then summing both equations we get that
    $$f(a+d)+e(b+c)+fp=f(b+c)+e(a+d)+ep$$
    Replacing $e=f+q$ on the right hand side of the equation we get that
    $$f(a+d)+e(b+c)+fp=f(b+c)+e(a+d)+fp+qp$$
    and by the Cancellation law we get that
    $$f(a+d)+e(b+c)=f(b+c)+e(a+d)+qp$$
    Which means that $f(b+c)+e(a+d)<f(a+d)+e(b+c)$ then $ae+bf+cf+de<af+be+ce+df$
    and thus $[(ae+bf,af+be)]<[(ce+df,cf+de)]$ therefore $xy<xz$.
    \item [(14)]
    Let's suppose that $\hat{0}=\hat{1}$ we want to arrive to a contradiction, then
    $[(1,1)]=[(1+1,1)]$ so $1+1=1+(1+1)$ and by the Cancellation law we have that
    $1=(1+1)$ which cannot be because there is no $a,b\in\N$ such that $a+b=1$.
    Therefore $\hat{0}\neq\hat{1}$.
    \end{itemize}
    \end{proof}
    \begin{proof}{\textbf{1.3.6}}
    Let us prove the rest of the Theorem 1.3.7 
    \begin{itemize}
    \item [\textbf{1.}] The function $i: \N \rightarrow \Z$ is injective.\\
    Let $i(n)=i(m)$ then by definition of $i$ we have that $[(n+1,1)]=[(m+1,1)]$ thus
    $(n+1)+1=1+(m+1)$ and by the Cancellation law we have that $n=m$.
    \item [\textbf{3.}] $i(1)=\hat{1}$ \\
    By definition of $i$ we have that $i(1)=[(1+1,1)]$ and therefore $i(1)=\hat{1}$.
    \item [\textbf{4b.}] $i(ab)=i(a)i(b)$\\
    By definition of $i$ we have that $i(ab)=[(ab+1,1)]$ then
    \begin{align*}
        i(ab) &= [(ab+1,1)] \\
              &= [(ab+a+b+1+1,a+b+1+1)] \\
              &= [((a+1)(b+1)+1,(a+1)+(b+1))] \\
              &= [(a+1,1)]\cdot [(b+1,1)] \\
              &= i(a)i(b)        
    \end{align*}
    \item [\textbf{4c.}] $a < b$ if and only if $i(a) < i(b)$.\\
    ($\rightarrow$) By definition $a<b$ means that $b=a+p$ where $p\in\N$ then applying
    the function $i$ to both sides of the equation we have that $i(b)=i(a+p)$ and
    because of what we have proven in \textbf{4a} we have that $i(b)=i(a)+i(p)$ which by
    definition means $i(a)< i(b)$.\\
    ($\leftarrow$) By definition $i(a)<i(b)$ means that $[(a+1,1)]<[(b+1,1)]$ and thus
    $(a+1)+1<1+(b+1)$ and because of the Cancellation law we have that $a<b$ as we
    wanted.
    \end{itemize}
    \end{proof}
    \begin{proof}{\textbf{1.3.7}}
    \begin{itemize}
        \item [(1)] ($\rightarrow$)
        Let $x < y$ then by the Addition Law for Order we have that $x + (-x) < y+(-x)$
        then by the Inverses Law for Addition we have that $0 < y+(-x)$ applying the
        Addition Law for Order again we have that $0 +(-y) < (y+(-x))+(-y)$ then by applying
        the Commutative and Associative Law for Addition we have that $-y < (y+(-y))+(-x)$
        and therefore $-y < -x$.\\
        ($\leftarrow$)
        Let $-y <-x$ then by the Addition Law for Order we have that $x + (-y) < x+(-x)$
        then by the Inverses Law for Addition we have that $x+(-y) < 0$ applying the
        Addition Law for Order again we have that $(x + (-y))+y < 0 + y$ then by applying
        the Commutative and Associative Law for Addition we have that $((-y)+y)+x < y$
        and therefore $x < y$.
        \item [(2)] ($\rightarrow$)
        Let $z<0$ and $x<y$ then $-z > 0$ because Lemma 1.4.5 part(8) and by the
        Multiplication Law for Order we have that $x(-z) < y(-z)$ which by the Lemma
        1.3.8 part 6 we know that $-xz < -yz$ thus $xz > yz$ because what we saw in part
        (1) of this problem.\\
        ($\leftarrow$)
        Let $xz>yz$ where $z <0$ because what we saw in part (1) of this problem this
        means that $-xz < -yz$ and then by the Multiplication Law for Order we have that
        $x<y$ because $-z > 0$.
    \end{itemize}
    \end{proof}
    \begin{proof}{\textbf{1.3.8}}
        From Theorem 1.3.9 we know that if $z \in \Z$ there is no $y \in \Z$ such that
        $z<y<z+1$ then there is no $x$ such that $0<x<1$ so if $x>0$ it must be that
        $x \geq 1$.\\
        If $x<0$ then $-x>0$ and as we saw this means that $-x\geq 1$ and by what we
        proved in problem 1.3.7 given that $-1<0$ then $-1 \cdot -x < -1 \cdot 1$
        therefore $-(1 \cdot (-x)) = -(-x) = x < -1$ where in the equalities we are
        using the fact that $(-x)y = -xy = x(-y)$.  
    \end{proof}
    \begin{proof}{\textbf{1.3.9}}
        \begin{itemize}
            \item [(1)]
            From the part (9) of the Lemma 1.4.5 we know that $0 < 1$ then by the
            Addition Law for Order we have that $0 + 1 < 1 + 1$ by the Identity Law for
            Addtion we have that $1 < 1+1$ and now let us call $2$ the following addition
            $2 = 1+1$ therefore $1 < 2$.
            \item [(2)]
            Suppose $2x = 1$ where $x \in \Z$ then as we proved in part (1) we have that
            $2x < 2 = 2 \cdot 1$ then by the Multiplication Law for Order we have that
            $x < 1$ then $x \leq 0$ if $x = 0$ then $2 \cdot 0 = 0 \neq 1$ so must
            happen that $x < 0$ then by Lemma 1.4.5 part (11) $2 \cdot x < 0$ but
            $1 > 0$ which is a contradiction. Therefore $2x \neq 1$. 
        \end{itemize}
    \end{proof}
    \begin{proof}{\textbf{1.3.10}}
        Let's define $G' = i^{-1}(G)$ which is the inverse image of $G$ then
        $G' \subseteq \N$ and because of the Well-Ordering Principle for $\N$ there is
        $m \in G'$ such that $m \leq g$ for all $g \in G'$. By appying $i$ to both sides
        of the inequality we have that $i(m) \leq i(g)$ which we can do because of part
        4c of Theorem 1.3.7.
        We know that $G \subseteq \{x \in \Z ~|~x>\hat{0}\} = i(\N)$ then the elements
        of $G$ have the form $i(a)$ where $a \in G'$ then $i(m),i(g) \in G$ so we want
        to check that $i(m)$ is the minimum then let's suppose that $i(m) > i(g)$ we
        want to arrive to a contradiction then $[(m+1,1)] > [(g+1,1)]$ and thus
        $m+1+1>g+1+1$ by the Cancellation law we have that $m>g$ which is a
        contradiction as we wanted because $m$ is the minimum of $G'$. Therefore
        $i(m) \leq i(g)$ for all $i(g) \in G$.         
    \end{proof}
    \begin{proof}{\textbf{1.3.11}}
        \begin{itemize}
        \item [(1)]
        By adding to both sides of the equation $(-z)$ we get that $(x + z) + (-z) = (y + z) + (-z)$
        and by the Associative Law for Addition we have that $x + (z + (-z)) = y + (z + (-z))$
        then because of the Inverse Law for Addition we have that $x + 0 = y + 0$
        which means that $x = y$ because of the Identity Law for Addition.
        \item [(3)]
        By the Inverses Law for Addition we know that $(x+y) + (-(x+y)) = 0$ then adding
        to both sides of the equation $-x$ and $-y$ we get that
        $(-x) + (-y) + (x+y) + (-(x+y)) = (-x) + (-y)$ then by using multiple times the
        Commutative and Associative Law for addition we get that
        $(x + (-x)) + (y + (-y)) + (-(x+y)) = (-x) + (-y)$ then
        $0 + 0 + (-(x+y)) = (-x) + (-y)$ because of the Inverses Law for Addition and
        therefore $-(x+y) = (-x) + (-y)$ because of the identity law for Addition.
        \item [(4)]
        Let us suppose that $x\cdot 0 \neq 0$ we want to arrive to a contradiction 
        then by the Trychotomy Law it must hold
        that $x \cdot 0 > 0$ or $x \cdot 0 < 0$. Let's suppose that $x \cdot 0 > 0$
        holds then by the Addition Law for Order we have that $(x \cdot 0) + x > 0 + x$
        and because the Identity Law for Multiplication we have that  
        $(x \cdot 0) + (x \cdot 1) > 0 + x$ then by the Identity Law for Addition and
        the Distributive Law we have that $x \cdot (0 + 1) > x$ again by the Identity
        Law for Addition we have that $x \cdot 1 > x$ then $x > x$ because of the
        Identity Law for Multiplication but this cannot be then it must hold that
        $x \cdot 0 < 0$ but by the same type of arguments we see that this cannot be
        either. Therefore it must be that $x \cdot 0 = 0$.
        \item [(5)]
        Suppose that $y = x + k$ where $k \in \Z$ then we get that $xz = yz = (x + k)z$
        and by adding to both sides of the equation $-xz$ we get that
        $xz + (-xz) = (xz + kz) + (-xz)$ then by applying the Associative and 
        Commutative Law for Addition we get that $xz + (-xz) = (xz + (-xz)) + kz$
        and by the Inveses Law for Addition we get that $0 = 0 + kz = kz$ then either
        $k = 0$ or $z = 0$ by the Non Zero Divisors Law, but we know that $z \neq 0$
        then it must be that $k = 0$ therefore $y = x + 0 = x$.
        \item [(7)]
        ($\rightarrow$) If $xy=1$ we have a few cases we need to address to prove that
        either $x=y=1$ or $x=y=-1$. Given that if $x >0$ and $y >0$ then $xy > 0$ and
        if $x>0$ and $y<0$ then $xy < 0$ we can rule a lot of cases by taking into
        account that either both $x$ and $y$ are positive or both are negative. Also
        the case where one of them or both are $0$ is also ruled out because of part (4)
        of this Lemma.
        \begin{itemize}
            \item Let's check first the case where both $x$ and $y$ are positive.
            Let then $x > 1$ and $y > 1$ by multiplying the $y$ inequality by $x$
            we have that $xy>x\cdot 1$ then because of the Identity Law for
            Multiplication we have that $xy > x > 1$ thus $xy \neq 1$.
            \item Now let $x<-1$ and $y<-1$ then $-x > 1 > 0$ because of part (8) of this
            Lemma so we multiply both sides of the $y$ inequality by $-x$ as
            $-xy < -1 \cdot 1$ and because of the Identity Law for Multiplication we
            have that $-xy < -1 < 0$ and then again by the part (8) of this Lemma we have
            that $xy > 1$ thus $xy \neq 1$.
            \item Finally, the only option left is that $x=y=1$ or $x=y=-1$. In the first
            case by the Identity Law for Multiplication we have that
            $xy = 1 \cdot 1 = 1$ and in the second case if $x=y=-1$ we have that
            $xy= (-1) \cdot (-1) = -((-1)\cdot 1)$ because of part (6) of this Lemma and
            then $xy = -(-1) = 1$ because the Identity Law for Multiplication and part
            (2) of this Lemma.            
        \end{itemize}
        Therefore, if $xy=1$ then $x=y=1$ or $x=y=-1$.\\ 
        ($\leftarrow$) As shown before if $x=y=1$ or $x=y=-1$ then $xy=1$.
        \item [(8)]
        ($\rightarrow$) If $x > 0$ then by adding to both sides of the equation $-x$ we
        have that $x+ (-x) > 0 + (-x)$ then because of the Identity Law for Addition we
        have that $x+ (-x) > -x$ and because of the Inverses Law for Addition $0 > -x$.\\
        ($\leftarrow$) If $-x < 0$ then by adding to both sides of the equation $x$ we
        have that $x+ (-x) < x + 0$ and because of the Identity Law and the Inverses Law
        for Addition we have that $0 < x$.\\
        ($\rightarrow$) If $x < 0$ then by adding to both sides of the equation $-x$ we
        have that $x+ (-x) < 0 + (-x)$ and because of the Identity Law and the Inverses
        Law for Addition we have that $0 < -x$.\\
        ($\leftarrow$) If $-x > 0$ then by adding to both sides of the equation $x$ we
        have that $x+ (-x) > x + 0$ and because of the Identity Law and the Inverses
        Law for Addition we have that $0 > x$.
\cleardoublepage
        \item [(10)]
        If $x \leq y$ then either $x = y$ or $x < y$ by definition and in the same way
        if $y \leq x$ then either $y = x$ or $y < x$.\\
        In case $x=y$ and $y=x$ then we are done.\\
        In case $x = y$ and $y < x$ then by replacing $y$ we have that $x<x$ which isn't
        true and thus $x = y$ must be true. The same can be proven for $y=x$ and $x<y$.\\
        In case $x < y$ and $y < x$ then by the Transitive Law $x<x$ which is not true
        and then must be that $x=y$.
        \item [(11)]
        If $x >0$ and $y>0$ then by multipyling the $y$ inequality by $x$ we have that
        $xy > x \cdot 0$ and we can do that because of the Theorem 1.3.5 part (13) and
        because of the result we proved in part (4) of this Lemma then $xy>0$.\\
        If $x >0$ and $y<0$ then by multipyling the $y$ inequality by $x$ we have that
        $xy < x \cdot 0$ and we can do that because of the Theorem 1.3.5 part (13) and
        because of the result we proved in part (4) of this Lemma then $xy < 0$.
    \end{itemize}
    \end{proof}
    \begin{proof}{\textbf{1.4.2}}
        Let $n \in \N$ also we know that $\N$ is defined as $\N = \{x \in \Z ~|~ x > 0\}$ then
        $n \in \Z$ and $n > 0$ by adding $1 > 0$ to both sides of the inequality we have
        that $n +1 > 0 +1 = 1$ where $0 +1 = 1$ because of the Identity Law for Addition
        and as we saw $1>0$ then $n+1>0$ and also $n+1 \in \Z$ therefore $n+1 \in \N$.  
    \end{proof}
    \begin{proof}{\textbf{1.4.3}}\\
        ($\rightarrow$) If $x,y\in\Z$ and $x \leq y$ then by definition either $x=y$ or
        $x<y$ if the last one holds then by adding $-x$ to both sides of the inequality
        we have that $x+(-x)<y+(-x)$ and because of the Inverses Law for Addition we
        have that $0 < y+(-x)$ and if we now add $-y$ to both sides of the inequality
        we have that $(-y)+0<(-y)+(y+(-x))$ and because of the Identity Law for Addition
        and the Associative Law we have that $-y < ((-y)+y)+(-x)$ then again by the
        Inverses Law for Addition we have that $-y < -x$.\\
        But if $x=y$ holds then applying the exact same steps as before we have that
        $-y=-x$.\\
        ($\leftarrow$) If $x,y\in\Z$ and $-y \leq -x$ then by definition either $-y=-x$ or
        $-y<-x$ if the last one holds then by adding $x$ to both sides of the inequality
        we have that $x+(-y)<x+(-x)$ and because of the Inverses Law for Addition we
        have that $x+(-y)< 0$ and if we now add $y$ to both sides of the inequality
        we have that $(x+(-y))+y<0+y$ and because of the Identity Law for Addition
        and the Associative Law we have that $x + ((-y)+y) < y$ then again by the
        Inverses Law for Addition we have that $x < y$.\\
        But if $-y=-x$ holds then applying the exact same steps as before we have that
        $x=y$.
    \end{proof}

\end{document}






















